\documentclass[english]{short-notes}

\title{Perverse sheaves and special subvarieties}
\author{Clemens Koppensteiner}

\usepackage{math-ag}

\addbibresource{global.bib}
%\bibliography{global.bib}

\newcommand\dualize{\mathbb D}
\newcommand\lc[1]{\Gamma_{\mkern-3mu#1}}

\begin{document}

\maketitle
\tableofcontents

\section{Setup and notation}

Let $X$ be a finite-dimensional Noetherian separated scheme over an algebraically closed field $k$.
We write $\O = \O_X$ for the structure sheaf of $X$.
Let $G$ be an algebraic group over $k$ acting on $X$.
For the moment we include the possibility of the action being trivial.
We write $X^{\mathrm{top}}$ for the subset of the topological space of $X$ consisting of generic points of $G$-invariant subschemes and equip $X^{\mathrm{top}}$ with the induced topology.

We write $D(X)$, $D_{qc}(X)$ and $D_c(X)$ for the derived category of $\O_X$-modules and its full subcategories consisting of complexes with quasi-coherent and coherent cohomology sheaves respectively.
The corresponding categories for $G$-equivariant sheaves (i.e.\ the categories for the quotient stack $[X/G]$) will be denoted $D(X^G)$, $D_{qc}(X^G)$ and $D_c(X^G)$.
As usual, $D^b(X)$ (etc.) will be the full subcategory of $D(X)$ consisting of complexes with cohomology in only finitely many degrees.
All functors will be derived, though we will usually drop the $R$ or $L$.

Let $Z$ be a closed subset of $X$.
For an $\O_X$-module $\sheaf F$ we let $\lc Z\sheaf F$ be the subsheaf of $\sheaf F$ of sections with support in $Z$ \cite[Varition~3 in IV.1]{Hartshorne:1966:ResiduesAndDuality}.
%If $\sheaf F$ is quasi-coherent, then by \cite[Theorem~V.4.1]{Hartshorne:1966:ResiduesAndDuality} we have
%\[
%\lc Z\sheaf F = \varinjlim_n \sheafHom_{\O_X}(\rquot{\O_X}{\sheaf I^n}, \sheaf F).
%\]
By abuse of notation, we will usually simply write $\lc Z$ for the right-derived functor $R\lc Z\colon D(X) → D(X)$.

Let $x$ be a (not necessarily closed) point of $x$ and $\sheaf F ∈ D^b(X)$.
Then $\mathbf ι_x^*\sheaf F = \sheaf F_x ∈ D^b(\catModules{\O_x})$ will denote the (derived) functor of talking stalks.
We further set $\mathbf ι_x^!\sheaf F = \mathbf ι_x^*\lc {\overline x}$, cf.~\cite[Varition~8 in IV.1]{Hartshorne:1966:ResiduesAndDuality}.

We assume that $X$ has a $G$-equivariant dualizing complex $\sheaf R$ (see \cite[Definition~1]{Bezrukavnikov:arXiv:PerverseCoherentSheaves}) which we assume to be normalized.
For $\sheaf F ∈ D(X)$ (or $D(X^G)$) we write $\dualize \sheaf F = \sheafHom_{\O_X}(\sheaf F,\sheaf R)$ for its dual.
Then,
\begin{itemize}
    \item the functor $\dualize$ takes $D^b_{c}(X)$ to $D^b_c(X)$ (resp.~$D_c^b(X^G)$ to $D_c^b(X^G)$);
    \item for $\sheaf F ∈ D^b_c(X)$ (or $D^b_c(X^G)$), the natural morphism $\sheaf F → \dualize\dualize F$ is an isomorphism;
    \item for $x ∈ X$, $\mathbf ι_x^!(\sheaf R)$ is concentrated in $-\dim x$ (where $\dim x \coloneq \dim\overline x$).
\end{itemize}

\section{Perverse coherent sheaves}
\label{sec:Kashiwara}%

Choose a decreasing function $p\colon \{0,\dotsc,\dim X\} → ℤ$, called perversity, such that $\overline p(n) = -n - p(n)$ is also decreasing.
For $x ∈ X^{\mathrm{top}}$ we set $p(x) = p(\dim x)$.
Then $p\colon X^{\mathrm{top}} → ℤ$ is a monotone and comonotone perversity in the sense of \cite[Definition~3]{Bezrukavnikov:arXiv:PerverseCoherentSheaves}, i.e.\ if $x' ∈ \overline x$, then $p(x') ≥ p(x)$ and $\overline p(x') ≥ \overline p(x)$.

Recall that Bezrukavnikov (following Deligne) \cite{Bezrukavnikov:arXiv:PerverseCoherentSheaves,ArinkinBezrukavnikov:arXiv:PerverseCoherentSheaves} defines a t-structure on $D_c^b(X^G)$ by taking the following full subcategories:
\begin{align*}
    \perv[p] D^{≤0}(X) & = 
    \{ \sheaf F ∈ D_c^b(X^G) : \mathbf ι_x^*\sheaf F ∈ D^{≤p(x)}(\catModules{\O_x}) \text{ for all $x ∈ X^{\mathrm{top}}$}\}, \\
    \perv[p] D^{≥0}(X) & = 
    \{ \sheaf F ∈ D_c^b(X^G) : \mathbf ι_x^!\sheaf F ∈ D^{≥p(x)}(\catModules{\O_x}) \text{ for all $x ∈ X^{\mathrm{top}}$}\}.
\end{align*}
The heart of this t-structure is called the category of perverse sheaves (with respect to the perversity $p$).

In \cite{Kashiwara:2004:tStructureOnHolonomicDModuleCoherentOModules}, Kashiwara also gives a definition of a perverse t-structure on $D^b_{coh}(X)$.
The two definitions match (up to a shift by $\dim X$), but there doesn't seem to be a reference for this is the literature.
So we will prove it in the following proposition:

\begin{Prop}
    \label{prop:equivDeligneKashiwara}%
    Let $p$ be a monotone and comonotone perversity function.
    Then,
    \begin{align*}
        \perv D^{≤0}(X) & = 
        \{ \sheaf F ∈ D_c^b(X^G) : p(\dim \supp H^{k}(\sheaf F)) ≥ k \text{ for all $k$}\}; \\
        \perv D^{≥0}(X) & = 
        \{ \sheaf F ∈ D_c^b(X^G) : \lc {\overline x}\sheaf F ∈ D^{≥p(x)}(X) \text{ for all $x ∈ X^{\mathrm{top}}$}\}.
    \end{align*}
\end{Prop}

\begin{proof}
    First let $\sheaf F ∈ \perv D^{≤0}(X)$ and assume for contradiction that there exists an integer $k$ such that $p(\dim \supp H^{k}(\sheaf F)) < k$.
    Let $x$ be the generic point of an irreducible component of $\supp H^{k}(\sheaf F)$.
    Then $\dim x = \dim \supp H^{k}(\sheaf F)$ and $H^k(\mathbf ι_x^* \sheaf F) \ne 0$.
    But $\mathbf ι_x^*\sheaf F ∈ D^{≤p(x)}(\catModules{\O_x})$, a contradiction.

    Conversely assume that $p(\dim \supp H^{k}(\sheaf F)) ≥ k$ for all $k$ and let $x ∈ X^{\mathrm{top}}$.
    We have $H^k(\mathbf ι_x^*\sheaf F) \ne 0$ if and only if $\overline x ⊆ \supp H^{k}(\sheaf F)$ (and hence $\dim x ≤ \dim \supp H^{k}(\sheaf F)$).
    Thus monotonicity of the perversity implies that $\sheaf F ∈ \perv D^{≤0}(X)$.

    Now assume that $\sheaf F ∈ \perv D^{≥0}(X)$.
    We proceed by induction on $\dim x$.
    If $\dim x = 0$, then $\mathbf ι_x^!\sheaf F = Γ(X,\lc {\overline x})$ and thus $\lc {\overline x}\sheaf F ∈ D^{≥p(x)}(X)$ by assumption.
    For the induction step there are two cases:
    If $\supp \lc {\overline x} \sheaf F = \overline x$, then $H^k(\mathbf ι_x^!\sheaf F) = H^k(\mathbf ι_x^*\lc {\overline x} \sheaf F)$ vanishes if and only if $H^k(\lc {\overline x})$ vanishes and we are done.
    If $\supp \lc {\overline x} \sheaf F ⊆ \overline x$ then the implication follows by induction and monotonicity of the perversity.

    The last implication follows directly from the relation $\mathbf ι_x^!\sheaf F = \mathbf ι_x^*\lc {\overline x} \sheaf F$.
\end{proof}

\begin{Prop}[{\cite[Proposition~4.3]{Kashiwara:2004:tStructureOnHolonomicDModuleCoherentOModules}}]
    \label{cor:dualStandard}
    The dualizing functor $\dualize_X$ sends the standard t-structure on $D_c^b(X^G)$ to the one associated to the perversity $p(n) = -\dim n$ (i.e.\ $p(x) = -\dim x$).
\end{Prop}

\begin{proof}
    The standard t-structure is given by the constant perversity $p(x) = 0$.
    Thus the statement follows immediately from $\dualize \perv[p] D^{≤0}(X) = \perv[\overline p] D^{≥0}(X)$ \cite[Lemma~5]{Bezrukavnikov:arXiv:PerverseCoherentSheaves}.
\end{proof}

\section{Measuring subvarieties}

From now on we will assume that the $G$-action has finitely many orbits.
If $U_{n}$ is the union of all $n$-dimensional orbits, then we assume that $\bigcup_{n < k} U_{n}$ is always closed\footnote{Is this always true? Is there a simple condition for this to be true?}.

\begin{Def}
    Let $p$ be a perversity.
    A \emph{$p$-measuring subvariety} of $X$ is an irreducible smooth subvariety $Z$ of $X$ such that $\dim(\overline x ∩ Z) = p(x) + \dim x$ for each $x ∈ X^{\mathrm{top}}$ with $\overline x ∩ Z \ne \emptyset$.

    We say that $X$ has \emph{enough $p$-measuring subvarieties} if for each $x ∈ X^{\mathrm{top}}$ there exists a $p$-measuring subvariety $Z$ with $Z ∩ \overline x \ne 0$.
\end{Def}

Note that comonotonicity of $p$ ensures that if $\dim y ≤ \dim x$ then $\dim (\overline y ∩ Z) ≤ \dim (\overline x ∩ Z)$ for each $p$-measuring $Z$.

\begin{Q}
    Under which conditions does $X$ have enough $p$-measuring subvarieties?
\end{Q}

Our main theorem will easily follow from the next two lemmas.

\begin{Lem}
    \label{lem:supportAndLocalCohomology-}%
    Let $\sheaf F ∈ \catCoh{X^G}$ be a $G$-equivariant coherent sheaf on $X$, let $p$ be a monotone perversity and let $n$ be an integer.
    Assume that $X$ has enough $p$-measuring subvarieties.
    Then the following are equivalent:
    \begin{enumerate}
        \item $p(\dim \supp \sheaf F) ≥ n$;
        \item $H^\ell(\lc Z\sheaf F) = 0$ for all $\ell ≥ -n+1$ and all measuring subvarieties $Z$.
    \end{enumerate}
\end{Lem}

Note that $H^\ell(\lc Z\sheaf F)$ means the $\ell$-th cohomology sheaf of the complex $R\lc Z\sheaf F$.

\begin{proof}
    First we will show that (i) implies to (ii).
    We will do so by induction on $d = \dim\supp\sheaf F$.
    The statement is trivial for $d = 0$ as then $\lc Z\sheaf F = \sheaf F$ or vanishes.

    Since $\supp \sheaf F$ is always a union of orbits, we can restrict to the support and assume that $\supp \sheaf F = X$.
    Let $U$ be the union of all $d$-dimensional orbits and let $Y$ be its complement.
    By assumption $U$ is open and $Y$ closed.
    Give $Y$ the reduced subscheme structure.
    From adjunction we get a surjective morphism $φ\colon \sheaf F → {ι_Y}_* {ι_Y}^* \sheaf F$ (here ${ι_Y}_*$ and ${ι_Y}^*$ are the underived $\O$-module operations, sending coherent sheaves to coherent sheaves).
    Let $\sheaf G$ be the kernel of this map, so that we get a short exact sequence
    \[
    0 → \sheaf G → \sheaf F → {ι_Y}_* {ι_Y}^* \sheaf F → 0
    \]
    in $\catCoh{X^G}$.
    Since $p$ is decreasing we have $p(\dim \supp {ι_Y}_* {ι_Y}^* \sheaf F) ≥ p(\dim \supp \sheaf F) ≥n$ so that $H^\ell(\lc Z{ι_Y}_* {ι_Y}^*\sheaf F) = 0$ for all $\ell ≥ -n+1$ by induction.
    So we only need to show that $H^\ell(\lc Z\sheaf G) = 0$ for all $\ell ≥ -n + 1$.
    Since $\sheaf G$ is supported on the open set $U$, we can restrict to $U$ and compute $\lc {Z∩U}\res{\sheaf G}U$.
    By regularity, $Z∩U$ is (locally) defined by $-p(\dim U)$ equations\footnote{Reference?}.
    Thus \cite[Theorem~3.3.1]{BrodmannSharp:1998:LocalCohomology} says that $H^\ell(\lc {Z∩U}\res{\sheaf G}U) = 0$ for $\ell > -p(\dim U)$ and hence in particular for all $\ell > -n$, as required.

    Now assume conversely that $H^\ell(\lc Z\sheaf F) = 0$ for all $\ell ≥ -n+1$ and all measuring subvarieties $Z$.
    We have to show that $p(\dim \supp \sheaf F) ≥ n$.
    Assume that this is not true and set $d = \dim \supp \sheaf F$.
    Hence $p(d) \lneq n$.
    As before, we can assume that $X = \supp \sheaf F$.
    Choose a $p$-measuring subvariety $Z$.
    Then $\codim_Z X = -p(d) ≥ -n + 1$.
    We can assume that $X$ is affine and write $X = \Spec A$.\footnote{There should be a reference for local cohomology of schemes where this assumption is not necessary.} 
    By flat base change,
    \[
    Γ(X,H^{-p(d)}(\lc Z \sheaf F))_{\ideal p} = 
    \left(H_{\ideal p}^{-p(d)}(Γ(X,\sheaf F))\right)_{\ideal p} =
    H_{\ideal p_{\ideal p}}^{-p(d)}(Γ(X,\sheaf F)_{\ideal p})
    \]
    By the Grothendieck non-vanishing theorem
    %\cite[Theorem~6.1.4]{BrodmannSharp:1998:LocalCohomology}
    \cite[Théorème~V.3.1]{SGA2}
    this is non-zero\footnote{Add an argument that $Γ(X,\sheaf F)_{\ideal p} \ne 0$ by equivariance.}, and hence $Γ(X,H^{-p(d)}(\lc Z \sheaf F)) \ne 0$, contradicting the assumption.
\end{proof}

\begin{Lem}[{\cite[Proposition~5.2]{Kashiwara:2004:tStructureOnHolonomicDModuleCoherentOModules}}]
    \label{lem:supportAndLocalCohomology+}%
    Let $\sheaf F ∈ D_c^b(X)$, $Z$ a closed subset of $X$ and $n$ an integer.
    Then $\lc Z\sheaf F ∈ D_{qc}^{≥n}(X)$ if and only if $-\dim(Z∩\supp(H^k(\dualize \sheaf F))) ≥ k + n$ for all $k$.
\end{Lem}

\begin{proof}
    The proof of \cite[Proposition~5.2]{Kashiwara:2004:tStructureOnHolonomicDModuleCoherentOModules} works for singular schemes as well --- just substitute $\dualize M$ for $M^*$ and Corollary \ref{cor:dualStandard} for \cite[Proposition~4.3]{Kashiwara:2004:tStructureOnHolonomicDModuleCoherentOModules}.
\end{proof}

\begin{Thm}
    Let $p$ be a monotone and comonotone perversity and assume that $X$ has enough $p$-measuring subvarieties.
    Let $\sheaf F ∈ D_c^b(X^G)$.
    Then $\sheaf F$ is perverse with respect to $p$ if and only if\/ $\lc Z\sheaf F$ is cohomologically concentrated in degree $0$ for each $p$-measuring subvariety $Z$.
    More precisely,
    \begin{enumerate}
        \item $\perv[p] D^{≤0}(X) = \{ \sheaf F ∈ D_c^b(X^G) : \lc Z\sheaf F ∈ D^{≤0}(X) \text{ for all $p$-measuring subvarieties $Z$}\}$;
        \item $\perv[p] D^{≥0}(X) = \{ \sheaf F ∈ D_c^b(X^G) : \lc Z\sheaf F ∈ D^{≥0}(X) \text{ for all $p$-measuring subvarieties $Z$}\}$.
    \end{enumerate}
\end{Thm}

\begin{proof}
    The first statement follows immediately from (i) and (ii), so we will prove those.
\begin{enumerate}
\item 
    We will use the description of $\perv[p] D^{≤0}(X)$ given by Proposition~\ref{prop:equivDeligneKashiwara}, i.e.
    \[
    \perv D^{≤0}(X) = \{ \sheaf F ∈ D_c^b(X^G) : p\left(\dim\left( \supp H^{n}(\sheaf F) \right)\right) ≥ n \text{ for all $n$}\}.
    \]
    We will use induction on the largest $k$ such that $H^k(\sheaf F) \ne 0$ to show that $\sheaf F ∈ \perv D^{≤0}$ if and only if $\lc Z\sheaf F ∈ D^{≤0}(X)$ for all $p$-measuring subvarieties $Z$.

    The equivalence is trivial for $k \ll 0$.
    For the induction step note that there is a distinguished triangle
    \[
    τ_{≤ k-1} \sheaf F → \sheaf F → H^k(\sheaf F)[-k] \xrightarrow{+1}.
    \]
    Applying the functor $\lc Z$ and taking cohomology we obtain 
    \begin{multline*}
        \cdots →
        H¹(\lc Z(τ_{≤ k-1} \sheaf F)) →
        H¹(\lc Z\sheaf F) →
        H^{k+1}(\lc Z(H^k(\sheaf F))) → \\
        H²(\lc Z(τ_{≤ k-1} \sheaf F)) →
        H²(\lc Z\sheaf F) →
        H^{k+2}(\lc Z(H^k(\sheaf F))) →
        \cdots.
    \end{multline*}
    By induction, $H^\ell(\lc Z(τ_{≤ k-1} \sheaf F))$ vanishes for $\ell ≥ 1$ so that $H^\ell(\lc Z\sheaf F) \cong H^{k+\ell}(\lc Z(H^k(\sheaf F)))$ for $\ell ≥ 1$.
    Thus the statement follows from Lemma~\ref{lem:supportAndLocalCohomology-}.
\item 
    By Proposition~\ref{prop:equivDeligneKashiwara} and Lemma~\ref{lem:supportAndLocalCohomology+}, $\sheaf F ∈ \perv D^{≥0}$ if and only if
    \begin{equation}
        \label{eq:main:+supp1}%
        \dim\left( \overline x ∩ \supp\left( H^k(\dualize F) \right) \right) ≤ -p(x) - k \quad \text{ for all $x ∈ X^{\mathrm{top}}$ and all $k$}.
    \end{equation}
    Using Lemma~\ref{lem:supportAndLocalCohomology+} for $\lc Z\sheaf F ∈ D^{≥0}(X)$, we see that we have to show the equivalence of \eqref{eq:main:+supp1} with
    \begin{equation}
        \label{eq:main:+supp2}%
        \dim\left( Z ∩ \overline x ∩ \supp\left( H^k(\dualize F) \right) \right) ≤ - k \quad \text{ for all $x ∈ X^{\mathrm{top}}$, $k$ and $p$-measuring $Z$}.
    \end{equation}
    Fix a $k$.
    Let us first show the implication from \eqref{eq:main:+supp1} to \eqref{eq:main:+supp2}.
    Since $H^k(\dualize \sheaf F)$ is $G$-equivariant, it suffices to show \eqref{eq:main:+supp2} assuming that $\overline x ⊆ \supp H^k(\dualize F)$.
    But then $\overline x ∩ \supp\left( H^k(\dualize F) \right) = \overline x$ and $Z ∩ \overline x ∩ \supp\left( H^k(\dualize F) \right) = Z ∩ \overline x$.
    Thus $\dim(Z ∩ \overline x) = -p(x) + \dim x ≤ -p(x) -k$, yields the implication.

    Conversely, assume that \eqref{eq:main:+supp2} holds.
    If $\overline x ⊆ \supp H^k(\dualize F)$, then the exact same argument as before shows that \eqref{eq:main:+supp1} holds for $x$.
    Otherwise $\overline x ∩ \supp\left( H^k(\dualize F) \right) = \overline y$ for some $y ∈ X^{\mathrm{top}}$ with $\dim y < \dim x$.
    Then \eqref{eq:main:+supp1} holds for $y$ and hence
    \[
    \dim\left( \overline x ∩ \supp\left( H^k(\dualize F) \right) \right) ≤
    -p(y) - k ≤
    -p(x) - k.
    \]
    by monotonicity of $p$.
    \qedhere
\end{enumerate}
\end{proof}

\begin{Ex}
    For the trivial perversity $p = 0$ we recover the standard definition of $\catCoh{X^G} \hookrightarrow D_c^b(X^G)$.
    For $p(n) = -n$ (i.e.\ $p(x) = -\dim x$), we recover the definition of Cohen-Macaulay sheaves.
\end{Ex}

\begin{Ex}
    Assume that all $G$-orbits are finite dimensional (e.g.\ when $X$ is the nilpotent cone of a semisimple complex Lie-algebra).
    Then we can define a \emph{middle perversity} $m$ by $m(x) = -\frac12 \dim x$.
    Note that $m = \overline m$ and hence $\dualize \perv[m] D^{≤0}(X) = \perv[m] D^{≥0}(X)$ \cite[Lemma~5(a)]{Bezrukavnikov:arXiv:PerverseCoherentSheaves}.
    In this case $m$-measuring subvarieties $Z$ are \enquote{universally half-dimensional} subvarieties, i.e.\ they intersect every orbit in a half-dimensional subvariety.
    In interesting cases, it should be possible to obtain such subvarieties as Lagrangians for (the resolution of) a symplectic structure on $X$.
\end{Ex}

\printbibliography

\end{document}
