\documentclass[english]{short-notes}

\title{Perverse sheaves and special subvarieties}
\author{Clemens Koppensteiner}

\usepackage{math-ag}

\addbibresource{global.bib}
%\bibliography{global.bib}

\newcommand\dualize{\mathbb D}
\newcommand\lc[1]{\Gamma_{\mkern-3mu#1}}

\begin{document}

\maketitle
\tableofcontents

\section{Setup and notation}

Let $X$ be a finite-dimensional Noetherian separated scheme over an algebraically closed field $k$.
Let $G$ be an algebraic group over $k$ acting on $X$.
For the moment we include the possibility of the action being trivial.
We write $X^{\mathrm{top}}$ for the subset of the topological space of $X$ consisting of generic points of $G$-invariant subschemes and equip $X^{\mathrm{top}}$ with the induced topology.

We write $D(X)$, $D_{qc}(X)$ and $D_c(X)$ for the derived category of $\O_X$-modules and its full subcategories consisting of complexes with quasi-coherent and coherent cohomology sheaves respectively.
The corresponding categories for $G$-equivariant sheaves (i.e.\ the categories for the quotient stack $[X/G]$) will be denoted $D(X^G)$, $D_{qc}(X^G)$ and $D_c(X^G)$.
As usual, $D^b(X)$ (etc.) will be the full subcategory of $D(X)$ consisting of complexes with cohomology in only finitely many degrees.
All functors will be derived, though we will usually drop the $R$ or $L$.

Let $Z$ be a closed subset of $X$.
For an $\O_X$-module $\sheaf F$ we let $\lc Z\sheaf F$ be the subsheaf of $\sheaf F$ of sections with support in $Z$ \cite[Varition~3 in IV.1]{Hartshorne:1966:ResiduesAndDuality}.
%If $\sheaf F$ is quasi-coherent, then by \cite[Theorem~V.4.1]{Hartshorne:1966:ResiduesAndDuality} we have
%\[
%\lc Z\sheaf F = \varinjlim_n \sheafHom_{\O_X}(\rquot{\O_X}{\sheaf I^n}, \sheaf F).
%\]
By abuse of notation, we will usually simply write $\lc Z$ for the right-derived functor $R\lc Z\colon D_{qc}(X) → D_{qc}(X)$.
Note that $\lc Z$ only depends on the closed subset $Z$, and not on the structure of $Z$ as a subscheme.

Let $x$ be a (not necessarily closed) point of $x$ and $\sheaf F ∈ D^b(X)$.
Then $\mathbf ι_x^*\sheaf F = \sheaf F_x ∈ D^b(\catModules{\O_x})$ will denote the (derived) functor of talking stalks.
We further set $\mathbf ι_x^!\sheaf F = \mathbf ι_x^*\lc {\overline x}$, cf.~\cite[Varition~8 in IV.1]{Hartshorne:1966:ResiduesAndDuality}.

We assume that $X$ has a $G$-equivariant dualizing complex $\sheaf R$ (see \cite[Definition~1]{Bezrukavnikov:arXiv:PerverseCoherentSheaves}) which we assume to be normalized.
For $\sheaf F ∈ D(X)$ (or $D(X^G)$) we write $\dualize \sheaf F = \sheafHom_{\O_X}(\sheaf F,\sheaf R)$ for its dual.
Then,
\begin{itemize}
    \item the functor $\dualize$ takes $D^b_{c}(X)$ to $D^b_c(X)$ (resp.~$D_c^b(X^G)$ to $D_c^b(X^G)$);
    \item for $\sheaf F ∈ D^b_c(X)$ (or $D^b_c(X^G)$), the natural morphism $\sheaf F → \dualize\dualize F$ is an isomorphism;
    \item for $x ∈ X$, $\mathbf ι_x^!(\sheaf R)$ is concentrated in degree $-\dim x$ (where $\dim x \coloneq \dim\overline x$).
\end{itemize}

\section{Perverse coherent sheaves}
\label{sec:Kashiwara}%

We fix a decreasing function $p\colon \{0,\dotsc,\dim X\} → ℤ$, called \emph{perversity}, such that $\overline p(n) \coloneq -n - p(n)$ is also decreasing.
For $x ∈ X^{\mathrm{top}}$ we set $p(x) = p(\dim x)$.
Then $p\colon X^{\mathrm{top}} → ℤ$ is a monotone and comonotone perversity in the sense of \cite[Definition~3]{Bezrukavnikov:arXiv:PerverseCoherentSheaves}, i.e.\ if $x' ∈ \overline x$, then $p(x') ≥ p(x)$ and $\overline p(x') ≥ \overline p(x)$.

Recall that Bezrukavnikov (following Deligne) defines a t-structure on $D_c^b(X^G)$ by taking the following full subcategories (see \cite{Bezrukavnikov:arXiv:PerverseCoherentSheaves,ArinkinBezrukavnikov:arXiv:PerverseCoherentSheaves}):
\begin{align*}
    \perv[p] D^{≤0}(X) & = 
    \{ \sheaf F ∈ D_c^b(X^G) : \mathbf ι_x^*\sheaf F ∈ D^{≤p(x)}(\catModules{\O_x}) \text{ for all $x ∈ X^{\mathrm{top}}$}\}, \\
    \perv[p] D^{≥0}(X) & = 
    \{ \sheaf F ∈ D_c^b(X^G) : \mathbf ι_x^!\sheaf F ∈ D^{≥p(x)}(\catModules{\O_x}) \text{ for all $x ∈ X^{\mathrm{top}}$}\}.
\end{align*}
The heart of this t-structure is called the category of perverse sheaves (with respect to the perversity $p$).

In \cite{Kashiwara:2004:tStructureOnHolonomicDModuleCoherentOModules}, Kashiwara also gives a definition of a perverse t-structure on $D^b_{c}(X)$.
The two definitions match (up to a shift by $\dim X$), but there doesn't seem to be a reference for this is the literature.
So we will prove it in the following proposition:

\begin{Prop}
    \label{prop:equivDeligneKashiwara}%
    Let $p$ be a monotone and comonotone perversity function.
    Then,
    \begin{align*}
        \perv D^{≤0}(X) & = 
        \{ \sheaf F ∈ D_c^b(X^G) : p(\dim \supp H^{k}(\sheaf F)) ≥ k \text{ for all $k$}\}; \\
        \perv D^{≥0}(X) & = 
        \{ \sheaf F ∈ D_c^b(X^G) : \lc {\overline x}\sheaf F ∈ D^{≥p(x)}(X) \text{ for all $x ∈ X^{\mathrm{top}}$}\}.
    \end{align*}
\end{Prop}

A crucial, but well-know, fact that we will implicitly use quite often in the following is that the support of a coherent sheaf is closed.
In particular, this means that if $x$ is a generic point and $\sheaf F$ a coherent sheaf, then $\mathbf ι_x^* \sheaf F = 0$ if and only if $\res{\sheaf F}U = 0$ for some open set $U$.

\begin{proof}
    First let $\sheaf F ∈ \perv D^{≤0}(X)$ and assume for contradiction that there exists an integer $k$ such that $p(\dim \supp H^{k}(\sheaf F)) < k$.
    Let $x$ be the generic point of an irreducible component of $\supp H^{k}(\sheaf F)$.
    Then $H^k(\mathbf ι_x^* \sheaf F) \ne 0$. 
    But on the other hand, $\mathbf ι_x^*\sheaf F ∈ D^{≤p(x)}(\catModules{\O_x})$ and $p(x) = p(\dim \supp H^{k}(\sheaf F)) < k$, yielding a contradiction.

    Conversely assume that $p(\dim \supp H^{k}(\sheaf F)) ≥ k$ for all $k$ and let $x ∈ X^{\mathrm{top}}$.
    If $H^k(\mathbf ι_x^*\sheaf F) \ne 0$, then $\dim x ≤ \dim \supp H^{k}(\sheaf F)$.
    Thus monotonicity of the perversity implies that $\sheaf F ∈ \perv D^{≤0}(X)$.

    Now assume that $\sheaf F ∈ \perv D^{≥0}(X)$.
    We want to show that $\lc {\overline x}\sheaf F ∈ D^{≥p(x)}(X)$ for all $x ∈ X^{\mathrm{top}}$.
    We will do so by induction on $\dim x$.
    If $\dim x = 0$, then $\mathbf ι_x^!\sheaf F = Γ(X,\lc {\overline x})$ and thus $\lc {\overline x}\sheaf F ∈ D^{≥p(x)}(X)$ by assumption.

    Now assume that the statement is true for $\dim x \le d$, i.e.\ $\lc {\overline x}\sheaf F ∈ D^{≥p(x)}(X)$ for all such $x$.
    In fact, using the Mayer-Vietoris sequence and monotonicity of the perversity function, we can actually assume that $\lc Y\sheaf F ∈ D^{≥p(\dim Y)}(X)$ for all closed subvarieties $Y \subseteq X$ with $\dim Y \le d$.

    Let $x ∈ X^{\mathrm{top}}$ be of dimension $d+1$.
    Let $k$ be the smallest integer such that $H^k(\lc {\overline x}\sheaf F) \ne 0$.
    Assume for contradiction that $k < p(x)$. 
    From \cite[Corollaire~VIII.2.2]{SGA2} and $\sheaf F ∈ \perv D^{≥0}(X)$ it follows that $H^k(\lc {\overline x}\sheaf F)$ is coherent.
    Set $Z = \supp H^k(\lc {\overline x}\sheaf F)$.
    Then, since $ι_x^* H^k(\lc {\overline x}\sheaf F)$ vanishes, $Z$ is a proper closed subset of $\overline x$.
    Consider the distinguished triangle
    \[
        H^k(\lc {\overline x}\sheaf F)[-k] →
        \lc {\overline x}\sheaf F →
        τ^{>k}\lc {\overline x}\sheaf F \xrightarrow{+1},
    \]
    and apply $\lc Z$ to it:
    \[
        H^k(\lc {\overline x}\sheaf F)[-k] →
        \lc Z \sheaf F →
        τ^{>k}\lc {Z}\sheaf F \xrightarrow{+1}.
    \]
    Since $\dim Z \le d$, we can use the induction hypothesis to deduce that $\lc Z \sheaf F$ is in degrees at least $p(\dim Z) \ge p(x) > k$, as is $τ^{>k}\lc {Z}\sheaf F$.
    Hence $H^k(\lc {\overline x}\sheaf F)$ has to vanish.

    The reverse implication follows directly from the relation $\mathbf ι_x^!\sheaf F = \mathbf ι_x^*\lc {\overline x} \sheaf F$.
\end{proof}

\begin{Prop}[{\cite[Proposition~4.3]{Kashiwara:2004:tStructureOnHolonomicDModuleCoherentOModules}}]
    \label{prop:dualStandard}
    The dualizing functor $\dualize_X$ sends the standard t-structure on $D_c^b(X^G)$ to the one associated to the perversity $p(n) = -\dim n$ (i.e.\ $p(x) = -\dim x$).
\end{Prop}

\begin{proof}
    The standard t-structure is given by the constant perversity $p(x) = 0$.
    Thus the statement follows immediately from $\dualize \perv[p] D^{≤0}(X) = \perv[\overline p] D^{≥0}(X)$ \cite[Lemma~5]{Bezrukavnikov:arXiv:PerverseCoherentSheaves}.
\end{proof}

\section{Measuring subvarieties}

From now on we will assume that the $G$-action has finitely many orbits.
If $U_{n}$ is the union of all $n$-dimensional orbits, then we assume that $\bigcup_{n < k} U_{n}$ is always closed\footnote{Is this always true? Is there a simple condition for this to be true?}.

\begin{Def}
    Let $p$ be a perversity.
    A \emph{$p$-measuring subvariety} of $X$ is an equidimensional subvariety $Z$ of $X$ such that the following conditions hold for each $x ∈ X^{\mathrm{top}}$ with $\overline x ∩ Z \ne \emptyset$:
    \begin{itemize}
        \item $\dim(\overline x ∩ Z) = p(x) + \dim x$;
        \item $\overline x ∩ Z$ is the underlying variety of a regularly embedded subscheme\footnote{I.e., up to radical it is locally defined by exactly $-p(x)$ functions.} in $\overline x$.
    \end{itemize}

    We say that $X$ has \emph{enough $p$-measuring subvarieties} if for each $x ∈ X^{\mathrm{top}}$ there exists a $p$-measuring subvariety $Z$ with $Z ∩ \overline x \ne \emptyset$.
\end{Def}

\begin{Rem}
    Let $Z$ be a $p$-measuring subvariety.
    Then $\dim(\overline x ∩ Z) = -\overline p(x)$.
    Thus comonotonicity of $p$ ensures that if $\dim y ≤ \dim x$ then $\dim (\overline y ∩ Z) ≤ \dim (\overline x ∩ Z)$ for each $p$-measuring $Z$.
    Monotonicity of $p$ then further says that $\dim (\overline x ∩ Z) - \dim (\overline y ∩ Z) ≤ \dim x - \dim y$.
    Further, we clearly have $0 \le \dim(\overline x ∩ Z) \le \dim x$ and hence $-\dim x \le p(x) \le 0$.
    We will show in Theorem~\ref{thm:existance} that these condition are actually sufficient for the existence of enough $p$-measuring subvarieties, at least when $X$ is affine.
\end{Rem}

Our main theorem will easily follow from the next two lemmas.

\begin{Lem}
    \label{lem:supportAndLocalCohomology-}%
    Let $\sheaf F ∈ \catCoh{X^G}$ be a $G$-equivariant coherent sheaf on $X$, let $p$ be a monotone perversity and let $n$ be an integer.
    Assume that $X$ has enough $p$-measuring subvarieties.
    Then the following are equivalent:
    \begin{enumerate}
        \item $p(\dim \supp \sheaf F) ≥ n$;
        \item $H^\ell(\lc Z\sheaf F) = 0$ for all $\ell ≥ -n+1$ and all measuring subvarieties $Z$.
    \end{enumerate}
\end{Lem}

Note that $H^\ell(\lc Z\sheaf F)$ means the $\ell$-th cohomology sheaf of the complex $R\lc Z\sheaf F$.

\begin{proof}
    Since $\supp \sheaf F$ is always a union of the closure of orbits, we can restrict to the support and assume that $\supp \sheaf F = X$.

    First assume that $p(\dim X) = p(\dim \supp \sheaf F) ≥ n$.
    By the definition of a $p$-measuring subvariety, this means that, up to radical, $Z$ can be locally defined by at most $-n$ equations.
    Thus $H^\ell(\lc Z\sheaf F) = 0$ for $\ell > -n$ \cite[Theorem~3.3.1]{BrodmannSharp:1998:LocalCohomology}. 

    Now assume conversely that $H^\ell(\lc Z\sheaf F) = 0$ for all $\ell ≥ -n+1$ and all measuring subvarieties $Z$.
    We have to show that $p(\dim \supp \sheaf F) ≥ n$.
    Set $d = \dim \supp \sheaf F$.
    Choose any $p$-measuring subvariety $Z$ that intersects $U$.
    Then $\codim_Z X = -p(d) ≥ -n + 1$.
    We will to show that $H^{-p(d)}(\lc Z \sheaf F) \ne 0$ and hence $p(d) \ge n$ by assumption.
    Take some affine open subset $U$ of $X$ such that $U \cap Z$ is irreducible in $U$.
    It suffices to show that the cohomology is non-zero in $U$.
    Thus we can assume without loss of generality that $X$ is affine, say $X = \Spec A$, and $Z$ is irreducible.
    Write $Z = V(\ideal p)$ for some prime ideal $\ideal p$ of $A$.
    By flat base change \cite[Theorem~4.3.2]{BrodmannSharp:1998:LocalCohomology},
    \[
    Γ(X,H^{-p(d)}(\lc Z \sheaf F))_{\ideal p} = 
    \left(H_{\ideal p}^{-p(d)}(Γ(X,\sheaf F))\right)_{\ideal p} =
    H_{\ideal p_{\ideal p}}^{-p(d)}(Γ(X,\sheaf F)_{\ideal p})
    \]
    Since $\dim_X \supp \sheaf F = \dim X = d$, the dimension of the $A_{\ideal p}$-module $Γ(X,\sheaf F)_{\ideal p}$ is $-p(d)$.
    Thus by the Grothendieck non-vanishing theorem
    \cite[Theorem~6.1.4]{BrodmannSharp:1998:LocalCohomology}
    %\cite[Théorème~V.3.1]{SGA2}
    $H_{\ideal p_{\ideal p}}^{-p(d)}(Γ(X,\sheaf F)_{\ideal p}) \ne 0$ and hence $Γ(X,H^{-p(d)}(\lc Z \sheaf F)) \ne 0$ as required.
\end{proof}

\begin{Lem}[{\cite[Proposition~5.2]{Kashiwara:2004:tStructureOnHolonomicDModuleCoherentOModules}}]
    \label{lem:supportAndLocalCohomology+}%
    Let $\sheaf F ∈ D_c^b(X)$, $Z$ a closed subset of $X$ and $n$ an integer.
    Then $\lc Z\sheaf F ∈ D_{qc}^{≥n}(X)$ if and only if $-\dim(Z∩\supp(H^k(\dualize \sheaf F))) ≥ k + n$ for all $k$.
\end{Lem}

\begin{proof}
    The proof of \cite[Proposition~5.2]{Kashiwara:2004:tStructureOnHolonomicDModuleCoherentOModules} works for singular schemes as well --- just substitute $\dualize M$ for $M^*$ and Proposition~\ref{prop:dualStandard} for \cite[Proposition~4.3]{Kashiwara:2004:tStructureOnHolonomicDModuleCoherentOModules}.
\end{proof}

We are now in a good position to prove the main theorem.

\begin{Thm}
    \label{thm:main}%
    Let $p$ be a monotone and comonotone perversity and assume that $X$ has enough $p$-measuring subvarieties.
    Let $\sheaf F ∈ D_c^b(X^G)$.
    Then $\sheaf F$ is perverse with respect to $p$ if and only if\/ $\lc Z\sheaf F$ is cohomologically concentrated in degree $0$ for each $p$-measuring subvariety $Z$.
    More precisely,
    \begin{enumerate}
        \item $\perv[p] D^{≤0}(X) = \{ \sheaf F ∈ D_c^b(X^G) : \lc Z\sheaf F ∈ D^{≤0}(X) \text{ for all $p$-measuring subvarieties $Z$}\}$;
        \item $\perv[p] D^{≥0}(X) = \{ \sheaf F ∈ D_c^b(X^G) : \lc Z\sheaf F ∈ D^{≥0}(X) \text{ for all $p$-measuring subvarieties $Z$}\}$.
    \end{enumerate}
\end{Thm}

\begin{proof}
    The first statement follows immediately from (i) and (ii), so we will prove those.
\begin{enumerate}
\item 
    We will use the description of $\perv[p] D^{≤0}(X)$ given by Proposition~\ref{prop:equivDeligneKashiwara}, i.e.
    \[
    \perv D^{≤0}(X) = \{ \sheaf F ∈ D_c^b(X^G) : p\left(\dim\left( \supp H^{n}(\sheaf F) \right)\right) ≥ n \text{ for all $n$}\}.
    \]
    We will use induction on the largest $k$ such that $H^k(\sheaf F) \ne 0$ to show that $\sheaf F ∈ \perv D^{≤0}$ if and only if $\lc Z\sheaf F ∈ D^{≤0}(X)$ for all $p$-measuring subvarieties $Z$.

    The equivalence is trivial for $k \ll 0$.
    For the induction step note that there is a distinguished triangle
    \[
    τ_{<k} \sheaf F → \sheaf F → H^k(\sheaf F)[-k] \xrightarrow{+1}.
    \]
    Applying the functor $\lc Z$ and taking cohomology we obtain an exact sequence
    \begin{multline*}
        \cdots →
        H¹(\lc Z(τ_{<k} \sheaf F)) →
        H¹(\lc Z\sheaf F) →
        H^{k+1}(\lc Z(H^k(\sheaf F))) → \\
        H²(\lc Z(τ_{<k} \sheaf F)) →
        H²(\lc Z\sheaf F) →
        H^{k+2}(\lc Z(H^k(\sheaf F))) →
        \cdots.
    \end{multline*}
    By induction, $H^\ell(\lc Z(τ_{<k} \sheaf F))$ vanishes for $\ell ≥ 1$ so that $H^\ell(\lc Z\sheaf F) \cong H^{k+\ell}(\lc Z(H^k(\sheaf F)))$ for $\ell ≥ 1$.
    Thus the statement follows from Lemma~\ref{lem:supportAndLocalCohomology-}.
\item 
    By Proposition~\ref{prop:equivDeligneKashiwara} and Lemma~\ref{lem:supportAndLocalCohomology+}, $\sheaf F ∈ \perv D^{≥0}$ if and only if
    \begin{equation}
        \label{eq:main:+supp1}%
        \dim\left( \overline x ∩ \supp\left( H^k(\dualize F) \right) \right) ≤ -p(x) - k \quad \text{ for all $x ∈ X^{\mathrm{top}}$ and all $k$}.
    \end{equation}
    Using Lemma~\ref{lem:supportAndLocalCohomology+} for $\lc Z\sheaf F ∈ D^{≥0}(X)$, we see that we have to show the equivalence of \eqref{eq:main:+supp1} with
    \begin{equation*}
        \dim\left( Z ∩ \supp\left( H^k(\dualize F) \right) \right) ≤ - k \quad \text{ for all $k$ and $p$-measuring $Z$}.
    \end{equation*}
    Since there are only finitely many orbits, this is in turn equivalent to
    \begin{equation}
        \label{eq:main:+supp2}%
        \dim\left( Z ∩ \overline x ∩ \supp\left( H^k(\dualize F) \right) \right) ≤ - k \quad \text{ for all $x ∈ X^{\mathrm{top}}$, $k$ and $p$-measuring $Z$}.
    \end{equation}
    We will show the equivalence for each fixed $k$ separately.
    Let us first show the implication from \eqref{eq:main:+supp1} to \eqref{eq:main:+supp2}.
    Since $H^k(\dualize \sheaf F)$ is $G$-equivariant and there are only finitely many $G$-orbits, it suffices to show \eqref{eq:main:+supp2} assuming that $\dim x \le \dim \supp H^k(\dualize F)$ and $\overline x \cap \supp H^k(\dualize F) \ne \emptyset$.
    Then $\dim\left(\overline x ∩ \supp\left( H^k(\dualize F) \right)\right) = \dim \overline x$.
    Thus,
    \begin{multline*}
        \dim\left(Z ∩ \overline x ∩ \supp\left( H^k(\dualize F) \right) \right) \le
        \dim(Z ∩ \overline x) =
        p(x) + \dim x = \\
        p(x) + \dim\left(\overline x ∩ \supp\left( H^k(\dualize F) \right)\right) \le
        p(x) - p(x) - k
        =k.
    \end{multline*}
    
    Conversely, assume that \eqref{eq:main:+supp2} holds for $k$.
    If $\overline x \cap \supp H^k(\dualize F) = \emptyset$, then \eqref{eq:main:+supp1} is trivially true.
    Otherwise choose a $p$-measuring $Z$ that intersects $\supp H^k(\dualize F)$.
    First assume that $\overline x$ is contained in $\supp H^k(\dualize F)$.
    Then
    \begin{multline*}
        \dim\left(\overline x ∩ \supp\left( H^k(\dualize F) \right)\right) =
        \dim x =
        -p(x) + \dim(Z ∩ \overline x) = \\
        -p(x) + \dim\left(Z ∩ \overline x ∩ \supp\left( H^k(\dualize F) \right) \right) \le
        -p(x) - k.
    \end{multline*}
    Otherwise $\overline x ∩ \supp\left( H^k(\dualize F) \right) = \overline y$ for some $y ∈ X^{\mathrm{top}}$ with $\dim y < \dim x$.
    Then \eqref{eq:main:+supp1} holds for $y$ and hence
    \[
    \dim\left( \overline x ∩ \supp\left( H^k(\dualize F) \right) \right) =
    \dim\left( \overline y ∩ \supp\left( H^k(\dualize F) \right) \right) ≤
    -p(y) - k ≤
    -p(x) - k
    \]
    by monotonicity of $p$.
    \qedhere
\end{enumerate}
\end{proof}

\begin{Ex}
    For the trivial perversity $p = 0$ we recover the standard definition of $\catCoh{X^G} \hookrightarrow D_c^b(X^G)$.
    For $p(n) = -n$ (i.e.\ $p(x) = -\dim x$), we recover the definition of Cohen-Macaulay sheaves.
\end{Ex}

\begin{Thm}
    \label{thm:existance}%
    Assume that $X$ is affine and the perversity $p$ satisfies $-n \le p(n) \le 0$ and is monotone and comonotone.
    Then $X$ has enough $p$-measuring subvarieties.
\end{Thm}

\begin{proof}
    Let $X = \Spec A$.
    We will use induction on dimension $d$.
    More precisely, we will induct on the following statement:
    \begin{quote}
        There exists a closed subvariety $Z_d$ of $X$ such that
        \begin{itemize}
            \item $Z_d$ intersects all $G$-orbits of dimension at most $d$;
            \item $Z_d$ fulfills the $p$-measurement conditions for all $x ∈ X^{\mathrm{top}}$ with $\dim x \le d$;
            \item the codimension of $Z_d$ in $\overline x$ is at most $-p(x)$ for all $x ∈ X^{\mathrm{top}}$ with $\dim x > d$.
        \end{itemize}
    \end{quote}
    We set $p(-1) = 0$.
    The statement is trivially true for $d = -1$, e.g.~take $Z = X$.
    Assume that the statement is true for some $d \ge -1$.
    We want to show it for $d+1 \le \dim X$.

    If $p(d+1) = p(d) - 1$, then the $Z$ constructed for $d$ will still work for $d+1$.
    So we only have to consider the case that $p(d+1) = p(d)$.
    Let $k^- \le d$ be the largest integer such that there exists $x ∈ X^{\mathrm{top}}$ with $\dim x = k^-$ (or $-1$ if no such $x$ exists).
    Set $S^- = \bigcup \{ x ∈ X^{\mathrm{top}} : \dim x = k^-\}$.
    Similarly, let $k^+ \ge d+1$ be the smallest integer such that there exists $x ∈ X^{\mathrm{top}}$ with $\dim x = k^+$ and set $S^+ = \bigcup \{ x ∈ X^{\mathrm{top}} : \dim x = k^-\}$.
    Since there are only finitely many orbits, the sets $S^-$ and $S^+$ are closed.
    Let $\ideal a^-$ and $\ideal a^+$ be the ideals of $A$ defining these subvarieties.
    Choose a non-constant function $f ∈ \ideal a^- \setminus \ideal a^+$, i.e.~a function that vanishes on all of $S^-$, but is not identically $0$ on $S^+$.
    Set $Z_{d+1} = Z_d \cap V(f)$.
    If $Z_{d+1}$ intersects each orbit of dimension $d+1$ (if there are any), we are done.
    Otherwise, in each orbit of dimension $d+1$ choose a closed subvariety $\tilde Z$ that is cut out buy $-p(d+1)$ functions and does not intersect any lower-dimensional orbit.
    Add all those $\tilde Z$ to $Z_{d+1}$.
\end{proof}

%\begin{Ex}
%    Assume that all $G$-orbits are finite dimensional (e.g.\ when $X$ is the nilpotent cone of a semisimple complex Lie-algebra).
%    Then we can define a \emph{middle perversity} $m$ by $m(x) = -\frac12 \dim x$.
%    Note that $m = \overline m$ and hence $\dualize \perv[m] D^{≤0}(X) = \perv[m] D^{≥0}(X)$ \cite[Lemma~5(a)]{Bezrukavnikov:arXiv:PerverseCoherentSheaves}.
%    In this case $m$-measuring subvarieties $Z$ are \enquote{universally half-dimensional} subvarieties, i.e.\ they intersect every orbit in a half-dimensional subvariety.
%    In interesting cases, it should be possible to obtain such subvarieties as Lagrangians for (the resolution of) a symplectic structure on $X$.
%\end{Ex}

\printbibliography

\end{document}
