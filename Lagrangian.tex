\documentclass[english,biblatex-alpha]{short-notes}

\title{Perverse coherent sheaves and special subvarieties}
\author{Clemens Koppensteiner}

\usepackage{math-ag}

\addbibresource{global.bib}
%\bibliography{global.bib}

\newcommand\dualize{\mathbb D}
\newcommand\lc[1]{\Gamma_{\mkern-3mu#1}}

\begin{document}

\maketitle
%\tableofcontents

\begin{abstract}
    Inspired by microlocal characterizations of constructible perverse sheaves we consider an alternative definition of perverse coherent sheaves as the category consisting of coherent sheaves $\sheaf F ∈ D^b_c(X^G)$ for which $RΓ_Z\sheaf F$ is concentrated in degree $0$ for a specific subvariety $Z$ of $X$.
\end{abstract}

\cgsIntro

\section{Introduction}

Consider a complex Kähler manifold $X$ and perverse sheaf (with respect to the middle perversity) $\sheaf F$ on $X$ and let $f\colon X → ℂ$ be a holomorphic function.
Then the vanishing cycle $φ_f(\sheaf F)$ is again perverse.
Assume that $f$ has a single critical point $p$ with $f(p) = 0$ and that the Hessian of $f$ is non-degenerate at $p$.
For such a function we can reinterpret the stalk at $p$ of the vanishing cycle measurement in the following way.

Let $C^\pm_{f,p}$ be the stable and unstable manifolds for the gradient of the Morse function $\Re f$.
These are Lagrangian submanifolds that intersect transversally in $p$.
Let $i_p$ be the inclusion of $p$ and let $j^\pm_{f,p}\colon C^\pm_{f,p} \hookrightarrow X$.
Then $i_p^*(j_{p,f}^+)^!\sheaf F$ and $i_p^!(j_{p,f}^-)^*\sheaf F$ are concentrated in degree $0$.

We would like to give an analogous characterization of perverse coherent sheaves (see Section~\ref{sec:Kashiwara} for a review of the theory of perverse coherent sheaves).
However on an arbitrary scheme it is not clear what \enquote{vanishing cycles of a holomorphic function} should be.
On the other hand, one could try to translate \enquote{Lagrangian submanifold} as half-dimensional closed subscheme.
It turns out that a statement like this can indeed be used to characterize perverse coherent sheaves.

Let $X$ be a variety with a group action so that all orbits are finite dimensional.
Then there exits a \enquote{middle perversity} on $[X/G]$.
We will define the notion of a \enquote{measuring subvariety} on $X$ (Definition~\ref{def:measuring}) and then show that a coherent sheaf $\sheaf F ∈ D^b_c(X^G)$ is perverse if and only if $RΓ_Z(\sheaf F)$ is concentrated in degree $0$ for all such measuring subvarieties (Theorem~\ref{thm:main}).
Thus it is possible to define the category of perverse coherent sheaves can be defined directly by the single type of measurement $RΓ_Z$, instead of having to define a t-structure first.


\subsection{Setup and notation}

Let $X$ be a finite-dimensional Noetherian separated scheme over an algebraically closed field $k$.
Let $G$ be an algebraic group over $k$ acting on $X$.
For the moment we include the possibility of the action being trivial.
We write $X^{\mathrm{top}}$ for the subset of the topological space of $X$ consisting of generic points of $G$-invariant subschemes and equip $X^{\mathrm{top}}$ with the induced topology.

We write $D(X)$, $D_{qc}(X)$ and $D_c(X)$ for the derived category of $\O_X$-modules and its full subcategories consisting of complexes with quasi-coherent and coherent cohomology sheaves respectively.
The corresponding categories for $G$-equivariant sheaves (i.e.\ the categories for the quotient stack $[X/G]$) will be denoted $D(X^G)$, $D_{qc}(X^G)$ and $D_c(X^G)$.
As usual, $D^b(X)$ (etc.) will be the full subcategory of $D(X)$ consisting of complexes with cohomology in only finitely many degrees.
All functors will be derived, though we will usually drop the $R$ or $L$.

\begin{cgs}
    Any basic introduction to derived categories will be sufficient in order to understand this article.
    The correct construction of the derived category of equivariant sheaves is somewhat involved, but not really necessary here, since all constructions will factor through the forgetful functor to non-equivariant category.
    The $G$-action is only used as a way to introduce something akin to a stratification on $X$.
    The equivariance condition forces the coherent sheaves to be vector bundles on each orbit and reduces the possible maps between sheaves so that the cone of a map is of that form (cf.~the introduction of \cite{ArinkinBezrukavnikov:arXiv:PerverseCoherentSheaves}).
\end{cgs}

Let $Z$ be a closed subset of $X$.
For an $\O_X$-module $\sheaf F$ we let $\lc Z\sheaf F$ be the subsheaf of $\sheaf F$ of sections with support in $Z$ \cite[Varition~3 in IV.1]{Hartshorne:1966:ResiduesAndDuality}.
%If $\sheaf F$ is quasi-coherent, then by \cite[Theorem~V.4.1]{Hartshorne:1966:ResiduesAndDuality} we have
%\[
%\lc Z\sheaf F = \varinjlim_n \sheafHom_{\O_X}(\rquot{\O_X}{\sheaf I^n}, \sheaf F).
%\]
By abuse of notation, we will usually simply write $\lc Z$ for the right-derived functor $R\lc Z\colon D_{qc}(X) → D_{qc}(X)$.
Note that $\lc Z$ only depends on the closed subset $Z$, and not on the structure of $Z$ as a subscheme.

Let $x$ be a (not necessarily closed) point of $x$ and $\sheaf F ∈ D^b(X)$.
Then $\mathbf ι_x^*\sheaf F = \sheaf F_x ∈ D^b(\catModules{\O_x})$ will denote the (derived) functor of talking stalks.
We further set $\mathbf ι_x^!\sheaf F = \mathbf ι_x^*\lc {\overline x}$, cf.~\cite[Varition~8 in IV.1]{Hartshorne:1966:ResiduesAndDuality}.

\begin{cgs}
    The original reference on local cohomology is \cite{SGA2}.
    A good English book about it is \cite{BrodmannSharp:1998:LocalCohomology}.
    Unfortunately, both texts don't use the formalism of derived categories and I don't know any comprehensive reference that develops local cohomology in that setting.
    However it should be pretty straightforward to translate the relevant results.
\end{cgs}

We assume that $X$ has a $G$-equivariant dualizing complex $\sheaf R$ (see \cite[Definition~1]{Bezrukavnikov:arXiv:PerverseCoherentSheaves}) which we assume to be normalized.
For $\sheaf F ∈ D(X)$ (or $D(X^G)$) we write $\dualize \sheaf F = \sheafHom_{\O_X}(\sheaf F,\sheaf R)$ for its dual.

\begin{cgs}
    Recall that the essential properties of the duality functor are:
    \begin{itemize}
        \item the functor $\dualize$ takes $D^b_{c}(X)$ to $D^b_c(X)$ (resp.~$D_c^b(X^G)$ to $D_c^b(X^G)$);
        \item for $\sheaf F ∈ D^b_c(X)$ (or $D^b_c(X^G)$), the natural morphism $\sheaf F → \dualize\dualize F$ is an isomorphism;
        \item for $x ∈ X$, $\mathbf ι_x^!(\sheaf R)$ is concentrated in degree $-\dim x$ (where $\dim x \coloneq \dim\overline x$).
    \end{itemize}
\end{cgs}

\section{Perverse coherent sheaves}
\label{sec:Kashiwara}%

A function $p\colon \{0,\dotsc,\dim X\} → ℤ$ will be called \emph{perversity}.
For $x ∈ X^{\mathrm{top}}$ we abuse notation and set $p(x) = p(\dim x)$.
Then $p\colon X^{\mathrm{top}} → ℤ$ is a perversity function in the sense of \cite{Bezrukavnikov:arXiv:PerverseCoherentSheaves}.
Note that with our definition we insist that $p(x)$ solely depends on the dimension of $\overline x$.
We will call a perversity \emph{monotone} if it is decreasing and \emph{comonotone} if the \emph{dual perversity} $\overline p(n) \coloneq -n - p(n)$ is decreasing.
It is \emph{strictly monotone} (resp.~\emph{strictly comonotone}) if for all $x,y ∈ X^{\mathrm{top}}$ with $\dim x < \dim y$ one has $p(x) > p(y)$ (resp.~$\overline p(x) > \overline p(y)$).
Note that a strictly monotone perversity is not necessarily strictly decreasing (e.g.~if $X$ only has even-dimensional orbits).

Recall that if $p$ is a monotone and comonotone perversity then Bezrukavnikov (following Deligne) defines a t-structure on $D_c^b(X^G)$ by taking the following full subcategories (see \cite{Bezrukavnikov:arXiv:PerverseCoherentSheaves,ArinkinBezrukavnikov:arXiv:PerverseCoherentSheaves}):
\begin{align*}
    \perv[p] D^{≤0}(X) & = 
    \{ \sheaf F ∈ D_c^b(X^G) : \mathbf ι_x^*\sheaf F ∈ D^{≤p(x)}(\catModules{\O_x}) \text{ for all $x ∈ X^{\mathrm{top}}$}\}, \\
    \perv[p] D^{≥0}(X) & = 
    \{ \sheaf F ∈ D_c^b(X^G) : \mathbf ι_x^!\sheaf F ∈ D^{≥p(x)}(\catModules{\O_x}) \text{ for all $x ∈ X^{\mathrm{top}}$}\}.
\end{align*}
The heart of this t-structure is called the category of perverse sheaves (with respect to the perversity $p$).

\begin{cgs}
    The original paper \cite{Bezrukavnikov:arXiv:PerverseCoherentSheaves} is easier to understand (as it doesn't use stacks). 
    However \cite{ArinkinBezrukavnikov:arXiv:PerverseCoherentSheaves} uses better notation (in particular the $ℂ$-module and $\O$-module restriction functors are notationally differentiated).
    So I'd recommend reading the two papers in parallel.
\end{cgs}

In \cite{Kashiwara:2004:tStructureOnHolonomicDModuleCoherentOModules}, Kashiwara also gives a definition of a perverse t-structure on $D^b_{c}(X)$.
While we work in Bezrukavnikov's setting (i.e.\ on the equivariant derived category on a potentially singular scheme), we need a description of the perverse t-structure that is closer to the one Kashiwara uses.

\begin{Prop}
    \label{prop:equivDeligneKashiwara}%
    Let $\sheaf F ∈ D_c^b(X^G)$ and let $p$ be a monotone and comonotone perversity function.
    \begin{enumerate}[(a)]
        \item
            The following are equivalent:
            \begin{enumerate}[(i)]
                \item $\sheaf F ∈ \perv D^{≤0}(X)$, i.e.\ $\mathbf ι_x^*\sheaf F ∈ D^{≤p(x)}(\catModules{\O_x})$ for all $x ∈ X^{\mathrm{top}}$;
                \item $p(\dim \supp H^{k}(\sheaf F)) ≥ k$ for all $k$.
            \end{enumerate}
        \item
            If $p$ is strictly monotone, then the following are equivalent
            \begin{enumerate}[(i)]
                \item $\sheaf F ∈ \perv D^{\ge 0}(X)$, i.e.\ $\mathbf ι_x^!\sheaf F ∈ D^{≥p(x)}(\catModules{\O_x})$ for all $x ∈ X^{\mathrm{top}}$;
                \item $\lc {\overline x}\sheaf F ∈ D^{≥p(x)}(X)$ for all $x ∈ X^{\mathrm{top}}$;
                \item $\lc {Y}\sheaf F ∈ D^{≥p(\dim Y)}(X)$ for all $G$-invariant equidimensional closed subsets $Z$ of $X$.
            \end{enumerate}
    \end{enumerate}
\end{Prop}

A crucial, but well-know, fact that we will implicitly use quite often in the following is that the support of a coherent sheaf is closed\textcgs{\ (e.g.~\cite[Exercise~II.5.6c]{Hartshorne:AG})}.
In particular, this means that if $x$ is a generic point and $\sheaf F$ a coherent sheaf, then $\mathbf ι_x^* \sheaf F = 0$ if and only if $\res{\sheaf F}U = 0$ for some open set $U \subseteq \overline x$.

\begin{proof}\leavevmode
    \begin{enumerate}[(a)]
        \item
            First let $\sheaf F ∈ \perv D^{≤0}(X)$ and assume for contradiction that there exists an integer $k$ such that $p(\dim \supp H^{k}(\sheaf F)) < k$.
            Let $x$ be the generic point of an irreducible component of $\supp H^{k}(\sheaf F)$.
            Then $H^k(\mathbf ι_x^* \sheaf F) \ne 0$. 
            But on the other hand, $\mathbf ι_x^*\sheaf F ∈ D^{≤p(x)}(\catModules{\O_x})$ and $p(x) = p(\dim \supp H^{k}(\sheaf F)) < k$, yielding a contradiction.

            Conversely assume that $p(\dim \supp H^{k}(\sheaf F)) ≥ k$ for all $k$ and let $x ∈ X^{\mathrm{top}}$.
            If $H^k(\mathbf ι_x^*\sheaf F) \ne 0$, then $\dim x ≤ \dim \supp H^{k}(\sheaf F)$.
            Thus monotonicity of the perversity implies that $\sheaf F ∈ \perv D^{≤0}(X)$.
        \item
            The implications from (iii) to (ii) and (ii) to (i) are trivial, so we only need to show that (i) implies (iii).
            So assume that $\sheaf F ∈ \perv D^{≥0}(X)$.
            We will induct on the dimension of $Y$.
            
            If $\dim Y = 0$, then $\mathbf ι_Y^!\sheaf F = Γ(X,\lc Y)$ and thus $\lc Y\sheaf F ∈ D^{≥p(x)}(X)$ by assumption.

            Now let $\dim Y = d > 0$.
            We will first assume that $Y$ is irreducible, i.e.\ $Y = \overline x$ for some $x ∈ X^{\mathrm{top}}$.
            Let $k < p(x)$.
            We have to show that $H^k(\lc {\overline x}\sheaf F) = 0$.

            We will first show that $H^k(\lc {\overline x}\sheaf F)$ is coherent.
            Let $j\colon X \setminus {\overline x} \hookrightarrow X$.
            Consider the distinguished triangle
            \[
                \lc {\overline x} \sheaf F → \sheaf F → j_*j^* \sheaf F \xrightarrow{+1}.
            \]
            Applying cohomology to it we get an exact sequence
            \[
                H^{k-1}(j_*j^*\sheaf F) → H^k(\lc{\overline x} \sheaf F) → H^k(\sheaf F).
            \]
            Now, $k-1 < p(x) - 1 \le p(x) - 2$, so that $H^{k-1}(j_*j^*\sheaf F)$ is coherent by the Grothendieck finiteness theorem in the form of \cite[Corollary~3]{Bezrukavnikov:arXiv:PerverseCoherentSheaves}.
            As $H^k(\sheaf F)$ is coherent by assumption, this implies that $H^k(\lc{\overline x} \sheaf F)$ also has to be coherent.

            Set $Z = \supp H^k(\lc {\overline x}\sheaf F)$.
            Then, since $ι_x^* H^k(\lc {\overline x}\sheaf F)$ vanishes, $Z$ is a proper closed subset of $\overline x$.
            Consider the distinguished triangle
            \[
                H^k(\lc {\overline x}\sheaf F)[-k] →
                \lc {\overline x}\sheaf F →
                τ^{>k}\lc {\overline x}\sheaf F \xrightarrow{+1},
            \]
            and apply $\lc Z$ to it:
            \[
                H^k(\lc {\overline x}\sheaf F)[-k] →
                \lc Z \sheaf F →
                τ^{>k}\lc {Z}\sheaf F \xrightarrow{+1}.
            \]
            Since $\dim Z < d$, we can use the induction hypothesis to deduce that $\lc Z \sheaf F$ is in degrees at least $p(\dim Z) \ge p(x) > k$, as is $τ^{>k}\lc {Z}\sheaf F$.
            Hence $H^k(\lc {\overline x}\sheaf F)$ has to vanish.

            If $Y$ is not irreducible, let $Y₁$ be an irreducible component of $Y$ and $Y₂$ be the union of the other components.
            Then Mayer-Vietoris gives a distinguished triangle
            \[
                \lc {Y₁\cap Y₂} \sheaf F → \lc {Y₁} \sheaf F \oplus \lc{Y₂}\sheaf F → \lc{Y} \sheaf F \xrightarrow{+1},
            \]
            where $\lc {Y₁\cap Y₂} \sheaf F ∈ D^{\ge p(\dim Y₁\cap Y₂)}(X) \subseteq D^{\ge p(\dim Y)+1}$ (by strict monotonicity of $p$) and $\lc{Y₁} \sheaf F$ and $\lc{Y₂} \sheaf F$ are in $D^{\ge p(\dim Y)}(X)$ by induction on the number of components of $Y$.
            Thus $\lc Y \sheaf F ∈ D^{\ge p(\dim Y)}$, as required.
            \qedhere
    \end{enumerate}
\end{proof}

\begin{Prop}[{\cite[Proposition~4.3]{Kashiwara:2004:tStructureOnHolonomicDModuleCoherentOModules}}]
    \label{prop:dualStandard}
    The dualizing functor $\dualize_X$ sends the standard t-structure on $D_c^b(X^G)$ to the one associated to the perversity $p(n) = -\dim n$ (i.e.\ $p(x) = -\dim x$).
\end{Prop}

\begin{proof}
    The standard t-structure is given by the constant perversity $p(x) = 0$.
    Thus the statement follows immediately from $\dualize \perv[p] D^{≤0}(X) = \perv[\overline p] D^{≥0}(X)$ \cite[Lemma~5]{Bezrukavnikov:arXiv:PerverseCoherentSheaves}.
\end{proof}

\section{Measuring subvarieties}

From now on we will assume that the $G$-action has finitely many orbits.

\begin{Def}
    \label{def:measuring}%
    Let $p$ be a perversity.
    A \emph{$p$-measuring subvariety} of $X$ is an equidimensional subvariety $Z$ of $X$ such that the following conditions hold for each $x ∈ X^{\mathrm{top}}$ with $\overline x ∩ Z \ne \emptyset$:
    \begin{itemize}
        \item $\dim(\overline x ∩ Z) = p(x) + \dim x$;
        \item $\overline x ∩ Z$ is the underlying variety of a regularly embedded subscheme\footnote{I.e., up to radical it is locally defined by exactly $-p(x)$ functions.} in $\overline x$.
    \end{itemize}

    We say that $X$ has \emph{enough $p$-measuring subvarieties} if for each $x ∈ X^{\mathrm{top}}$ there exists a $p$-measuring subvariety $Z$ with $Z ∩ \overline x \ne \emptyset$.
\end{Def}

\begin{Rem}
    Let $Z$ be a $p$-measuring subvariety.
    Then $\dim(\overline x ∩ Z) = -\overline p(x)$.
    Thus comonotonicity of $p$ ensures that if $\dim y ≤ \dim x$ then $\dim (\overline y ∩ Z) ≤ \dim (\overline x ∩ Z)$ for each $p$-measuring $Z$.
    Monotonicity of $p$ then further says that $\dim (\overline x ∩ Z) - \dim (\overline y ∩ Z) ≤ \dim x - \dim y$.
    Further, we clearly have $0 \le \dim(\overline x ∩ Z) \le \dim x$ and hence $-\dim x \le p(x) \le 0$.
    We will show in Theorem~\ref{thm:existance} that these condition are actually sufficient for the existence of enough $p$-measuring subvarieties, at least when $X$ is affine.
\end{Rem}

Our main theorem will easily follow from the next two lemmas.

\begin{Lem}
    \label{lem:supportAndLocalCohomology-}%
    Let $\sheaf F ∈ \catCoh{X^G}$ be a $G$-equivariant coherent sheaf on $X$, let $p$ be a monotone perversity and let $n$ be an integer.
    Assume that $X$ has enough $p$-measuring subvarieties.
    Then the following are equivalent:
    \begin{enumerate}
        \item $p(\dim \supp \sheaf F) ≥ n$;
        \item $H^\ell(\lc Z\sheaf F) = 0$ for all $\ell ≥ -n+1$ and all measuring subvarieties $Z$.
    \end{enumerate}
\end{Lem}

Note that $H^\ell(\lc Z\sheaf F)$ means the $\ell$-th cohomology sheaf of the complex $R\lc Z\sheaf F$.
\textcgs{Also note that the lemma is only interesting for $n<0$.}

\begin{proof}
    Since $\supp \sheaf F$ is always a union of the closure of orbits, we can restrict to the support and assume that $\supp \sheaf F = X$.

    First assume that $p(\dim X) = p(\dim \supp \sheaf F) ≥ n$.
    By the definition of a $p$-measuring subvariety, this means that, up to radical, $Z$ can be locally defined by at most $-n$ equations.
    Thus $H^\ell(\lc Z\sheaf F) = 0$ for $\ell > -n$ \cite[Theorem~3.3.1]{BrodmannSharp:1998:LocalCohomology}. 

    Now assume conversely that $H^\ell(\lc Z\sheaf F) = 0$ for all $\ell ≥ -n+1$ and all measuring subvarieties $Z$.
    We have to show that $p(\dim \supp \sheaf F) ≥ n$.
    Set $d = \dim \supp \sheaf F$.
    Choose any $p$-measuring subvariety $Z$ that intersects $U$.
    Then $\codim_Z X = -p(d) ≥ -n + 1$.
    We will to show that $H^{-p(d)}(\lc Z \sheaf F) \ne 0$ and hence $p(d) \ge n$ by assumption.
    Take some affine open subset $U$ of $X$ such that $U \cap Z$ is irreducible in $U$.
    It suffices to show that the cohomology is non-zero in $U$.
    Thus we can assume without loss of generality that $X$ is affine, say $X = \Spec A$, and $Z$ is irreducible.
    Write $Z = V(\ideal p)$ for some prime ideal $\ideal p$ of $A$.
    By flat base change \cite[Theorem~4.3.2]{BrodmannSharp:1998:LocalCohomology},
    \[
    Γ(X,H^{-p(d)}(\lc Z \sheaf F))_{\ideal p} = 
    \left(H_{\ideal p}^{-p(d)}(Γ(X,\sheaf F))\right)_{\ideal p} =
    H_{\ideal p_{\ideal p}}^{-p(d)}(Γ(X,\sheaf F)_{\ideal p})
    \]
    Since $\dim_X \supp \sheaf F = \dim X = d$, the dimension of the $A_{\ideal p}$-module $Γ(X,\sheaf F)_{\ideal p}$ is $-p(d)$.
    Thus by the Grothendieck non-vanishing theorem
    \cite[Theorem~6.1.4]{BrodmannSharp:1998:LocalCohomology}
    %\cite[Théorème~V.3.1]{SGA2}
    $H_{\ideal p_{\ideal p}}^{-p(d)}(Γ(X,\sheaf F)_{\ideal p}) \ne 0$ and hence $Γ(X,H^{-p(d)}(\lc Z \sheaf F)) \ne 0$ as required.
\end{proof}

\begin{Lem}[{\cite[Proposition~5.2]{Kashiwara:2004:tStructureOnHolonomicDModuleCoherentOModules}}]
    \label{lem:supportAndLocalCohomology+}%
    Let $\sheaf F ∈ D_c^b(X)$, $Z$ a closed subset of $X$ and $n$ an integer.
    Then $\lc Z\sheaf F ∈ D_{qc}^{≥n}(X)$ if and only if $-\dim(Z∩\supp(H^k(\dualize \sheaf F))) ≥ k + n$ for all $k$.
\end{Lem}

\textcgs{An analogous lemma holds in the topological situations, see~\cite[Exercise~X.10]{KashiwaraSchapira:1994:SheavesOnManifolds}.}

\begin{proof}
    The proof of \cite[Proposition~5.2]{Kashiwara:2004:tStructureOnHolonomicDModuleCoherentOModules} works for singular schemes as well --- just substitute $\dualize M$ for $M^*$ and Proposition~\ref{prop:dualStandard} for \cite[Proposition~4.3]{Kashiwara:2004:tStructureOnHolonomicDModuleCoherentOModules} and note that the dual standard perversity is strictly monotone so that Proposition~\ref{prop:equivDeligneKashiwara} can be applied.
\end{proof}

We are now in a good position to prove the main theorem.

\begin{Thm}
    \label{thm:main}%
    Let $p$ be a strictly monotone and (not necessarily strictly) comonotone perversity and assume that $X$ has enough $p$-measuring subvarieties.
    Let $\sheaf F ∈ D_c^b(X^G)$.
    Then $\sheaf F$ is perverse with respect to $p$ if and only if\/ $\lc Z\sheaf F$ is cohomologically concentrated in degree $0$ for each $p$-measuring subvariety $Z$.
    More precisely,
    \begin{enumerate}
        \item $\perv[p] D^{≤0}(X) = \{ \sheaf F ∈ D_c^b(X^G) : \lc Z\sheaf F ∈ D^{≤0}(X) \text{ for all $p$-measuring subvarieties $Z$}\}$;
        \item $\perv[p] D^{≥0}(X) = \{ \sheaf F ∈ D_c^b(X^G) : \lc Z\sheaf F ∈ D^{≥0}(X) \text{ for all $p$-measuring subvarieties $Z$}\}$.
    \end{enumerate}
\end{Thm}

\begin{proof}
    The first statement follows immediately from (i) and (ii), so we will prove those.
\begin{enumerate}
\item 
    We will use the description of $\perv[p] D^{≤0}(X)$ given by Proposition~\ref{prop:equivDeligneKashiwara}, i.e.
    \[
    \perv D^{≤0}(X) = \left\{ \sheaf F ∈ D_c^b(X^G) : p\left(\dim\left( \supp H^{n}(\sheaf F) \right)\right) ≥ n \text{ for all $n$}\right\}.
    \]
    We will use induction on the largest $k$ such that $H^k(\sheaf F) \ne 0$ to show that $\sheaf F ∈ \perv D^{≤0}$ if and only if $\lc Z\sheaf F ∈ D^{≤0}(X)$ for all $p$-measuring subvarieties $Z$.

    The equivalence is trivial for $k \ll 0$.
    For the induction step note that there is a distinguished triangle
    \[
    τ_{<k} \sheaf F → \sheaf F → H^k(\sheaf F)[-k] \xrightarrow{+1}.
    \]
    Applying the functor $\lc Z$ and taking cohomology we obtain an exact sequence
    \begin{multline*}
        \cdots →
        H¹(\lc Z(τ_{<k} \sheaf F)) →
        H¹(\lc Z\sheaf F) →
        H^{k+1}(\lc Z(H^k(\sheaf F))) → \\
        H²(\lc Z(τ_{<k} \sheaf F)) →
        H²(\lc Z\sheaf F) →
        H^{k+2}(\lc Z(H^k(\sheaf F))) →
        \cdots.
    \end{multline*}
    By induction, $H^\ell(\lc Z(τ_{<k} \sheaf F))$ vanishes for $\ell ≥ 1$ so that $H^\ell(\lc Z\sheaf F) \cong H^{k+\ell}(\lc Z(H^k(\sheaf F)))$ for $\ell ≥ 1$.
    Thus the statement follows from Lemma~\ref{lem:supportAndLocalCohomology-}.
\item 
    By Proposition~\ref{prop:equivDeligneKashiwara} and Lemma~\ref{lem:supportAndLocalCohomology+}, $\sheaf F ∈ \perv D^{≥0}$ if and only if
    \begin{align}
        \label{eq:main:+supp1}%
        & \dim\left( \overline x ∩ \supp\left( H^k(\dualize F) \right) \right) ≤ -p(x) - k &  \text{for all $x ∈ X^{\mathrm{top}}$ and all $k$}. \\
        %
        \intertext{Using Lemma~\ref{lem:supportAndLocalCohomology+} for $\lc Z\sheaf F ∈ D^{≥0}(X)$, we see that we have to show the equivalence of \eqref{eq:main:+supp1} with}
        %
        \notag
        & \dim\left( Z ∩ \supp\left( H^k(\dualize F) \right) \right) ≤ - k & \quad \text{ for all $k$ and $p$-measuring $Z$}. \\
        %
        \intertext{Since there are only finitely many orbits, this is in turn equivalent to}
        %
        \label{eq:main:+supp2}%
        & \dim\left( Z ∩ \overline x ∩ \supp\left( H^k(\dualize F) \right) \right) ≤ - k & \text{ $\forall\, x ∈ X^{\mathrm{top}}$, $k$ and $p$-measuring $Z$}.
    \end{align}
    We will show the equivalence for each fixed $k$ separately.
    Let us first show the implication from \eqref{eq:main:+supp1} to \eqref{eq:main:+supp2}.
    Since $H^k(\dualize \sheaf F)$ is $G$-equivariant and there are only finitely many $G$-orbits, it suffices to show \eqref{eq:main:+supp2} assuming that $\dim x \le \dim \supp H^k(\dualize F)$ and $\overline x \cap \supp H^k(\dualize F) \ne \emptyset$.
    Then $\dim\left(\overline x ∩ \supp\left( H^k(\dualize F) \right)\right) = \dim \overline x$.
    Thus,
    \begin{multline*}
        \dim\left(Z ∩ \overline x ∩ \supp\left( H^k(\dualize F) \right) \right) \le
        \dim(Z ∩ \overline x) =
        p(x) + \dim x = \\
        p(x) + \dim\left(\overline x ∩ \supp\left( H^k(\dualize F) \right)\right) \le
        p(x) - p(x) - k
        =k.
    \end{multline*}
    
    Conversely, assume that \eqref{eq:main:+supp2} holds for $k$.
    If $\overline x \cap \supp H^k(\dualize F) = \emptyset$, then \eqref{eq:main:+supp1} is trivially true.
    Otherwise choose a $p$-measuring $Z$ that intersects $\supp H^k(\dualize F)$.
    First assume that $\overline x$ is contained in $\supp H^k(\dualize F)$.
    Then
    \begin{multline*}
        \dim\left(\overline x ∩ \supp\left( H^k(\dualize F) \right)\right) =
        \dim x =
        -p(x) + \dim(Z ∩ \overline x) = \\
        -p(x) + \dim\left(Z ∩ \overline x ∩ \supp\left( H^k(\dualize F) \right) \right) \le
        -p(x) - k.
    \end{multline*}
    Otherwise $\overline x ∩ \supp\left( H^k(\dualize F) \right) = \overline y$ for some $y ∈ X^{\mathrm{top}}$ with $\dim y < \dim x$.
    Then \eqref{eq:main:+supp1} holds for $y$ and hence
    \[
    \dim\left( \overline x ∩ \supp\left( H^k(\dualize F) \right) \right) =
    \dim\left( \overline y ∩ \supp\left( H^k(\dualize F) \right) \right) ≤
    -p(y) - k ≤
    -p(x) - k
    \]
    by monotonicity of $p$.
    \qedhere
\end{enumerate}
\end{proof}

\begin{Ex}
    For the dual standard perversity $p(n) = -n$ (i.e.\ $p(x) = -\dim x$), we recover the definition of Cohen-Macaulay sheaves.
\end{Ex}

\begin{Thm}
    \label{thm:existance}%
    Assume that $X$ is affine and the perversity $p$ satisfies $-n \le p(n) \le 0$ and is monotone and comonotone.
    Then $X$ has enough $p$-measuring subvarieties.
\end{Thm}

\begin{proof}
    Let $X = \Spec A$.
    We will use induction on dimension $d$.
    More precisely, we will induct on the following statement:
    \begin{quote}
        There exists a closed equidimensional subvariety $Z_d$ of $X$ such that for all $x ∈ X^{\mathrm{top}}$ the following holds:
        \begin{itemize}
            \item $Z_d \cap \overline x \ne \emptyset$ and $Z_d \cap \overline x$ is regularly embedded in $\overline x$;
            \item if $\dim x \le d$, then $\dim(\overline x ∩ Z_d) = p(x) + \dim x$;
            \item if $\dim x > d$, then $\dim(\overline x ∩ Z_d) = p(d) + \dim x$.
        \end{itemize}
    \end{quote}
    We set $p(-1) = 0$.
    The statement is trivially true for $d = -1$, e.g.~take $Z = X$.
    Assume that the statement is true for some $d \ge -1$.
    We want to show it for $d+1 \le \dim X$.

    If $p(d) = p(d+1)$, then $Z_{d+1} = Z_{d}$ works.
    Otherwise, by (co)monotonicity $p(d+1) = p(d) - 1$.
    Set $S = \bigcup \{ \overline x ∈ X^{\mathrm{top}} : \dim x \le d\}$.
    Since there are only finitely many orbits, we can choose a function $f$ \textcgs{(by choosing individual $f_i$ and multiplying them)} such that $f$ vanishes identically on $S$, $V(f)$ does not share a component with $Z_d$ and $V(f)$ intersects every $\overline x$ with $\dim x > d$.
    Then $Z_{d+1} = Z_d \cap V(f)$ satisfies the conditions.
\end{proof}

%\begin{Ex}
%    Assume that all $G$-orbits are finite dimensional (e.g.\ when $X$ is the nilpotent cone of a semisimple complex Lie-algebra).
%    Then we can define a \emph{middle perversity} $m$ by $m(x) = -\frac12 \dim x$.
%    Note that $m = \overline m$ and hence $\dualize \perv[m] D^{≤0}(X) = \perv[m] D^{≥0}(X)$ \cite[Lemma~5(a)]{Bezrukavnikov:arXiv:PerverseCoherentSheaves}.
%    In this case $m$-measuring subvarieties $Z$ are \enquote{universally half-dimensional} subvarieties, i.e.\ they intersect every orbit in a half-dimensional subvariety.
%    In interesting cases, it should be possible to obtain such subvarieties as Lagrangians for (the resolution of) a symplectic structure on $X$.
%\end{Ex}

\printbibliography

\end{document}
