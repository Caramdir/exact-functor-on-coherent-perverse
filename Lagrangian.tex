\documentclass[english]{short-notes}

\title{Perverse sheaves and Lagrangians}

\usepackage{math-ag}

\addbibresource{global.bib}
%\bibliography{global.bib}

\newcommand\me{\normalcolor}
\newcommand\david{\color{black!60!green}}
\let\David\david

\begin{document}

\maketitle

\section{Traditional setting}

\david
Let $X$ be an $n$-dimensional complex smooth Kähler manifold.
Let $p$ be a point in $X$.
Let $f\colon X → ℂ$ be a \emph{generic} \emph{holomorphic} Morse function (note that everything is local, so that we can assume that non-constant holomorphic functions exist), such that
\begin{itemize}
    \item $f(p) = 0$;
    \item $\res{df}p = 0$;
    \item $\res{Hf}p $ is non-degenerate \me($Hf$ is the Hessian of $f$)\david.
\end{itemize}
Then there exist local coordinates such that $f \sim z₁² + z₂² + \dotsb + z_n² + (\text{h.o.\ terms})$.

Consider $\Re f\colon X → ℝ$,
\[
\Re f \sim x₁² + \dotsb + x_n² - y₁² - \dotsb -y_n² + (\text{h.o.\ terms}).
\]
Let $C_{f,p}^-$, $C_{f,p}^+$ be cells of flow lines for the gradient of $\Re f$ with respect to the Kähler metric.
\me(I assume that these are the same as the stable and unstable manifolds for the gradient of $\Re f$.)
\David
Note that $C_{f,p}^-$, $C_{f,p}^+$ are $n$-dimensional real manifolds, transversely intersecting at $p$.

\begin{Prop}
    $C_{f,p}^-$, $C_{f,p}^+$ are Lagrangian with respect to the Kähler form.
\end{Prop}

Let $i_p$ denote the inclusion of $p$ into $C_{f,p}^-$ and $C_{f,p}^+$.
Let $j_{p,f}^\pm\colon C_{f,p}^\pm \hookrightarrow X$ be the inclusions.

\begin{Thm}
    Let $\sheaf F$ be a bounded complex of sheaves with constructible cohomology.
    Then the following are equivalent:
    \begin{enumerate}
        \item $\sheaf F$ is perverse;
        \item For all $p ∈ X$ and $f$ as above, $i_p^* {j_{p,f}^+}^! \sheaf F$ and $i_p^!{j_{p,f}^-}^* \sheaf F$ are in degree $0$.
    \end{enumerate}
\end{Thm}

Note that the roles of $+$ and $-$ can be reversed by multiplying $f$ by $\i$.

\begin{Q}
    Does a similar statement hold for any smooth Lagrangian?
\end{Q}

\me

\section{Coherent setting}

Let $X$ be a complex variety.
Until I understand how symplectic resolutions work, I will assume that $X$ is smooth and hence has the structure of a symplectic manifold.
Let $G$ be an algebraic group acting on $X$ and assume that the action has finitely many orbits, all of which are even-dimensional.
We will write $D_c^b(X)$ for the bounded derived category of $G$-equivariant sheaves on $X$ with coherent cohomology (which is equivalent to $D^b_c([X/G])$).
The word \enquote{perverse sheaf} will mean an element of the heart of $D_c^b(X)$ with respect to the t-structure define by the middle perversity.

\begin{Def}
    Let $\sheaf F ∈ D_c^b(X)$.
    The \emph{Lagrangian measurements} of $\sheaf F$ is the data consisting of $i_p^*i_L^! \sheaf F$ and $i_p^!i_L^! \sheaf F$ for each inclusion $i_L\colon L \hookrightarrow X$ of a Lagrangian into $X$ and each point $i_p\colon p \hookrightarrow L$ of $L$.
    The element $\sheaf F$ is said to satisfy the \emph{Lagrangian measurement condition} (\emph{LMC}) if all Lagrangian measurements of $\sheaf F$ are cohomologically concentrated in degree $0$.
\end{Def}

\begin{Conjecture}
    An element $\sheaf F ∈ D_c^b(X)$ perverse if and only if it satisfies the Lagrangian measurement condition.
\end{Conjecture}

\subsection{Some lemmas}

The following lemma is mentioned in some note from a discussion with David.
I should find a reference (or prove it).

\david
\begin{Lem}[Poincaré lemma for !-restriction]
    Let $X$ be a smooth complex variety and $ι_Y\colon Y \hookrightarrow X$ a smooth subvariety.
    Then
    \[
    ι_Y^!\O_X \cong \res{\O_X}Y \otimes_{\O_X} Ω_{Y/X}^{\mathrm{top}}[-\codim_X Y].
    \]
\end{Lem}
\me

\begin{proof}
    ?
\end{proof}

From this it follows that if $\sheaf F$ is a locally free sheaf on $X$, then $ι_Y^!\sheaf F$ is concentrated in degree $\codim_X Y$.

\begin{Lem}
    Let $\sheaf F ∈ D_c^{\le 0}(X)$ and let $ι_Y\colon Y \hookrightarrow X$ a smooth subvariety.
    Then $H^i(ι_Y^!\sheaf F) = 0$ for $i \ge \codim_X Y + 1$.
\end{Lem}

\begin{Lem}
    Let $\sheaf F ∈ D_c^{\le 0}(X)$ such that $H^0(\sheaf F)$ is locally free and nowhere vanishing.
    Let $ι_Y\colon Y \hookrightarrow X$ a smooth subvariety.
    Then $H^{\codim_X Y}(ι_Y^! \sheaf F) \ne 0$.
\end{Lem}

\begin{proof}
    Represent $\sheaf F$ by a complex of locally free sheaves that is $0$ in positive degrees.
    Let $u\colon \sheaf F → H^0(\sheaf F)$ be the quotient map.
    We obtain a distinguished triangle
    \[
    ι_Y^!\sheaf F → ι_Y^! H^0(\sheaf F) → ι_Y^!\cone(u).
    \]
    From this we get the exact sequence
    \[
    H^{\codim_X Y}(ι_Y^!\sheaf F) → H^{\codim_X Y}(ι_Y^! H^0(\sheaf F)) → H^{\codim_X Y}(ι_Y^!\cone(u)).
    \]
    Note that $\cone(u) ∈ D_c^{\le -2}(X)$ so that the last term vanishes by the preceding lemma.
    By the Poincaré Lemma, the second term is non-zero, so that the first term must also be non-zero.
\end{proof}

\subsection{The two-dimensional case}

In this subsection we will prove the most basic case of the conjecture.
Let $X$ be smooth two-dimensional complex variety and $G$ act on it with one open orbit $U$ and a finite number of fixed points.
Since all questions are local, we can assume that there is only one fixed point, which we will call $0$.
We will denote the inclusion of $U$ into $X$ by $j$.
The perversity is then given by $p(U) = -1$ and $p(0) = 0$.

\subsubsection{Perverse sheaves satisfy LMC}

It suffices to show this for irreducible perverse sheaves.
These are described explicitly by \cite[Proposition~4.11]{ArinkinBezrukavnikov:arXiv:PerverseCoherentSheaves} (or in simpler terms by \cite[Corollary~4]{Bezrukavnikov:arXiv:PerverseCoherentSheaves}; note that there is a sign error in both of these statements).

An irreducible perverse sheaf $\sheaf F$ coming from the fixed point $0$ is a skyscraper sheaf at $0$ in degree $0$.
Hence it satisfies LMC.

Now let $\sheaf F = j_{!*}(L[1])$ for an irreducible $G$-equivariant vector bundle $L$.
Let $p ∈ U$ and $L$ be a Lagrangian through $p$.
As $\sheaf F$ is concentrated in degree $-1$ and is locally free on $U$, $i_L^!\sheaf F$ is concentrated in degree $0$ by the Poincaré Lemma, and hence so is $i_p^*i_L^! \sheaf F$.
TODO: $i_p^!i_L^*\sheaf F$.

Now let $p = 0$.
TODO: Prove this case.

\subsubsection{LMC-sheaves are perverse}

We want to show that if $\sheaf F ∈ D_c^b(X)$ satisfies LMC, then it is perverse.

Let $U$ be the open orbit, Lagrangian in $U$.
For all $p ∈ L$, the complex $i_p^*i_L^!\sheaf F$ is concentrated in degree $0$.
Hence $i_L^!\sheaf F$ is in degree $0$.
Thus $\sheaf F$ is in degrees $-1$ and $0$.
But since it has locally free cohomology, it cannot be in degree $0$ as otherwise $i_L^!\sheaf F$ would be in degree $1$.

Now let $L$ be a Lagrangian through $0$.
Then for all $p ∈ L$ the complex $i_p^* i_L^!\sheaf F$ is in degree $0$.
Hence $i_L^! \sheaf F$ is also in degree $0$ and thus $i_0^! \sheaf F = i_0^!i_L^! \sheaf F$ is in degrees $\ge 0$.

Finally we have to show that $i_0^* \sheaf F$ is in degrees $\le 0$.
By the preceding arguments, we know that $i_p^*i_L^*\sheaf F$ is concentrated in degree $-1$ for $p \ne 0$.
Inductively applying the following lemma to $\sheaf G = \i_L^*\sheaf F$ we conclude that $i_0^*\sheaf F = i_0^* \sheaf G$ is in degrees $\le 0$.

\begin{Lem}
    Let $L$ be a curve, $0 ∈ L$ a point and $\sheaf G ∈ D^{\le n}(L)$ ($n \ge 1$) so that the following conditions hold:
    \begin{enumerate}
        \item $\sheaf G_p$ is (cohomologically) in degree $-1$ for all $p \ne 0$.
        \item $i_0^! \sheaf G$ is (cohomologically) in degree $0$.
    \end{enumerate}
    Then $\sheaf G ∈ D^{\le n-1}(L)$.
\end{Lem}

\begin{proof}
    Let $\sheaf G$ be represented by the complex
    \[
    \sheaf G^\cx \colon \cdots → \sheaf G^{n-2} → \sheaf G^{n-1} → \sheaf G^{n} → 0.
    \]
    Note that by the assumptions $H^n(\sheaf G^\cx)$ and $H^{n-1}(\sheaf G^\cx)$ are skyscraper sheaves at $0$.
    Let $u \colon \sheaf G^\cx → H^n(\sheaf G^\cx)[-n]$ be the quotient map.
    The cone of this map is given by $\cone(u) = \sheaf G^\cx[1] \oplus H^n(\sheaf G^\cx)[-n]$ and has no cohomology in degrees $\ge n-1$.
    Applying $i_0^!$ we obtain a distinguished triangle
    \[
    i_0^!\sheaf G → \i_0^! H^n(\sheaf G)[-n] → i_0^! \cone(u)
    \]
    which yields an exact sequence (note that the second term is a skyscraper)
    \[
    H^n(i_0^!\sheaf G) → H^n(\sheaf G)_0 → H^n(i_0^! \cone(u)).
    \]
    Since the first term is $0$ by assumption, it suffices to prove that the third one is $0$ as well.

    Let $\sheaf H = \cone(u)$ be represented by the complex
    \[
    \sheaf H^\cx \colon \cdots → \sheaf H^{n-4} → \sheaf H^{n-3} → \sheaf H^{n-2} → 0.
    \]
    We have $H^i(\sheaf H) = H^{i+1}(\sheaf G)$ for $i \le n-2$.
    In particular $H^{n-2}(\sheaf H)$ is a skyscraper at $0$.
    Consider $v\colon \sheaf H^\cx → H^{n-2}(\sheaf H)[-n+2]$.
    Then $H^i(\cone(v)) = 0$ for $i \ge n-3$.
    The distinguished triangle
    \[
    i_0^!\sheaf H^\cx → i_0^!H^{n-2}(\sheaf H)[-n+2] → i_0^!\cone(v)
    \]
    yields an exact sequence
    \[
    H^{n-1}(\cone(v)) → H^{n}(i_0^!\sheaf H) → H^n(i_0^!H^{n-2}(\sheaf H)[-n+2]).
    \]
    The last term is vanishes as $i_0^!H^{n-2}(\sheaf H)[-n+2]$ is in degree $n-2$.
    By \cite[Proposition~3.2.2~(i)]{KashiwaraSchapira:1994:SheavesOnManifolds} every sheaf on a complex curve admits a $c$-soft resolution of length at most $2$.
    Thus the first term (and hence also the middle term) vanishes as well.
\end{proof}

\section{A t-structure?}

If the direct approach doesn't work, maybe I can prove at least some of the following conjectures.

\begin{Def}
    We define two full subcategories $\perv[L] D^{\le 0}$ and $\perv[L] D^{\ge 0}$ of $D_ℂ^b(X)$ by
    \begin{align*}
        \perv[L] D^{\le 0} & = \{ \sheaf F ∈ D_ℂ^b(X) : ι_p^!ι_L^*\sheaf F ∈ D^{\le 0}(ℂ_p) \text{ for all Lagrangians $L$ and all $p ∈ L$} \}, \\
        \perv[L] D^{\ge 0} & = \{ \sheaf F ∈ D_ℂ^b(X) : ι_p^*ι_L^!\sheaf F ∈ D^{\ge 0}(ℂ_p) \text{ for all Lagrangians $L$ and all $p ∈ L$} \}. 
    \end{align*}
\end{Def}

\begin{Conjecture}
    The pair $(\perv[L] D^{\le 0}, \perv[L] D^{\ge 0})$ forms a t-structure on $D_ℂ^b(X)$.
\end{Conjecture}

\begin{Conjecture}
    The t-structures $(\perv D^{\le 0}, \perv D^{\ge 0})$ and $(\perv[L] D^{\le 0}, \perv[L] D^{\ge 0})$ coincide.
\end{Conjecture}

\begin{Conjecture}
    An object $\sheaf F ∈ D_ℂ^b(X)$ is in $\perv[L] D^{\le 0} \cap \perv[L] D^{\ge 0}$ if and only if 
    both $ι_p^!ι_L^*\sheaf F $ and $ι_p^*ι_L^!\sheaf F$ are (cohomologically) concentrated in degree $0$ for all Lagrangians $L$ and all $p ∈ L$.
\end{Conjecture}

\printbibliography

\end{document}
