\documentclass[english]{short-notes}

\usepackage{math-ag}

\addbibresource{global.bib}
%\bibliography{global.bib}

\title{Perverse coherent sheaves}
\subtitle{Notes for oral exam}
\author{Clemens Koppensteiner}

\newcommand\derived{\mathbf D}
\newcommand\derivedcoh{\derived_{\mathrm{coh}}}
%\renewcommand\cat{\mathscr}
\let\setset\cover
\newcommand\inv{\mathrm{inv}}

\begin{document}

\renewcommand\top{\mathrm{top}}
\renewcommand\dual{\mathbb D}

\maketitle

\tableofcontents

\section{Introduction}

$X$, $X^\top$, $G$.

\section{Preliminaries}

\subsection{Category theory}

\subsubsection{Derived categories}\label{sec:derived_categories}

\begin{Def}
    additive category,
    Abelian category,
    homotopy category of complexes,
    derived category
\end{Def}

Derived category and roofs.

\subsubsection{Triangulated categories}\label{sec:triangulated_categories}

\begin{Def}
    A \emph{triangulated category} consists of an additive category $\cat D$ together with
    \begin{itemize}
        \item an automorphism (or auto-equivalence) $T\colon \cat D → \cat D$, called the \emph{translation functor}, and
        \item a collection of sequences
            \[ X \xrightarrow{u} Y \xrightarrow{v} Z \xrightarrow{w} TX, \]
            called \emph{distinguished} triangles,
    \end{itemize}
    satisfying the following axioms, where we write $X[1] \coloneq TX$,
        \begin{enumerate}
            \item[(TR~1)]
                Any triangle (i.e.\ sequence $X → Y → Z → X[1]$) isomorphic to a distinguished triangle is distinguished.
                Every morphism $u\colon X → Y$ is contained in a distinguished triangle $X \xrightarrow{u} Y → Z → X[1]$.
                Every triangle of the form $X \xrightarrow{\id} X → 0 → X[1]$ is distinguished.
            \item[(TR~2; rotation)]
                A triangle $X \xrightarrow{u} Y \xrightarrow{v} Z \xrightarrow{w} X[1]$ is distinguished, if and only if $Y \xrightarrow{v} Z \xrightarrow{w} X[1] \xrightarrow{-u[1]} Y[1]$ is distinguished.
            \item[(TR~3; morphisms)] 
                If $X \xrightarrow{u} Y \xrightarrow{v} Z \xrightarrow{w} X[1]$ and
                $X' \xrightarrow{u'} Y' \xrightarrow{v'} Z' \xrightarrow{w'} X'[1]$
                are distinguished, then for each morphism $(f,g)\colon u → u'$ there exists a morphism $h\colon Z → Z'$ such that $(f,g,h)$ is a morphism of triangles (where a morphism of triangles is defined as the obvious commutative diagram).
            \item[(TR~4; octahedral axiom)]
                If $X \xrightarrow{u} Y \xrightarrow{u} Z' → X[1]$,
                $Y \xrightarrow{v} Z \xrightarrow{} X' \xrightarrow{j} Y[1]$, and
                $X \xrightarrow{w} Z \xrightarrow{} Y' \xrightarrow{} X[1]$
                are distinguished triangles such that $w = v∘u$, then there exist morphisms $f\colon Z' → Y'$ and $g\colon Y' → X'$ such that
                \begin{enumerate}[1)]
                    \item $(\id[X],v,f)$ is a morphism of triangles;
                    \item $(u, \id[Z],g)$ is a morphism of triangles;
                    \item $Z' \xrightarrow{f} Y' \xrightarrow{g} X' \xrightarrow{i[1]} Z'[1]$ is a distinguished triangle.
                \end{enumerate}
    \end{enumerate}
    A \emph{functor of triangulated categories} is a functor of the underlying categories which commutes with the translation functors and sends distinguished triangles to distinguished triangles.
\end{Def}

\begin{Ex}
    Let $\cat A$ be an additive category and $\mathbf K\cat A$ the category of complexes of objects of $\cat A$ and $\Hom_{\mathbf K\cat A}(K,L)$ the set of homotopy classes of morphisms of complexes from $K$ to $L$.
    Let $TK = K[1]$ be defined by $(K[1])^n = K^{n+1}$ (with a sign change in the differential).
    In $\mathbf K\cat A$ the distinguished triangles are gives by $K \xrightarrow{u} L → \operatorname{cone}(u) → K[1]$, giving $\mathbf K\cat A$ the structure of a triangulated category.
    Here we need to work in the homotopy category of complexes, as otherwise the triangle $K \xrightarrow{\id} K → 0 → K[1]$ wouldn't be distinguished (the mapping cone of $\id$ is only quasi-isomorphic to 0).
    Also note that if $u\colon K → L$ is injective, then $\operatorname{cone}(u) \cong L/K$, so that every short exact sequence of complexes is a distinguished triangle.

    If $\cat A$ is Abelian, this structure descends to the derived category $\derived \cat A$.
\end{Ex}

\begin{Def}
    Let $\cat D$ be a triangulated category and $\cat A$ an Abelian category.
    A functor $F\colon \cat D → \cat A$ is called \emph{cohomological} if it maps each distinguished triangle $X → Y → Z → X[1]$ to an exact sequence $F(X) → F(Y) → F(Z)$.
\end{Def}

\begin{Ex}
    For any $X ∈ \cat D$ the functors $\Hom_{\cat D}(X,\cdot)$ and $\Hom_{\cat D}(\cdot, X)$  are cohomological (the second one being defined on the opposite triangulated category (with translation functor the inverse of $T$ and reading distinguished triangles the other way)).
    Hence, if $F\colon \cat D → \cat A$ is cohomological, it induces a long exact sequence
    \[
    \dotsc → FX → FY → FZ → FTX → FTY → \dotsc
    \]
    for each distinguished triangle $X → Y → Z → X[1]$ (using TR~2 repeatedly).
\end{Ex}

\subsubsection{t-categories}\label{sec:t-categories}

\begin{Def}
    A \emph{t-category} is a triangulated category $\cat D$ together with a pair $(\cat D^{≤0},\cat D^{≥0})$ of full subcategories, called a \emph{t-structure}, such that, writing $\cat D^{≤n} = \cat D^{≤0}[n]$ and $\cat D^{≥n} = \cat D^{≥0}[n]$, the following conditions hold:
    \begin{enumerate}
        \item if $X ∈ \cat D^{≤0}$ and $Y ∈ \cat D^{≥1}$, then $\Hom_{\cat D}(X,Y) = 0$;
        \item $\cat D^{≤0} ⊆ \cat D^{≤1}$ and $\cat D^{≥0} ⊇ \cat D^{≥1}$;
        \item for each $X ∈ \cat D$ there exists a distinguished triangle $A → X → B → A[1]$ such that $A ∈ \cat D^{≤0}$ and $B ∈ \cat D^{≥1}$.
    \end{enumerate}

    The full subcategory $\cat D^{≤0} ∩ \cat D^{≥0}$ is called the \emph{heart} of the t-structure, sometimes denoted $\cat D^\heartsuit$.
\end{Def}

\begin{Ex}
    If $\cat A$ is an Abelian category, then its derived category $\derived \cat A$ admits a standard t-structure given by
    \begin{align*}
        \derived \cat A^{≤0} &= \{ F ∈ \derived\cat A : H^j(F) = 0 \text{ for all } j > 0 \}, \\
        \derived \cat A^{≥0} &= \{ F ∈ \derived\cat A : H^j(F) = 0 \text{ for all } j < 0 \}.
        \qedhere
    \end{align*}
\end{Ex}

\begin{Prop}
    The inclusions $\cat D^{≤n} → \cat D$ (resp.~$\cat D^{≥n} → \cat D$) admit right (resp.~left) adjoint functors $τ^{≤n}\colon \cat D → \cat D^{≤n}$ (resp.\ $τ^{≥n}\colon \cat D → \cat D^{≥n}$), called the \emph{truncation functors}.
\end{Prop}

These functors $τ^{≤0}$ and $τ^{≥1}$ are given by the third property of the definition of a t-structure. 
The other truncation functors are just translates.

\begin{Ex}
    For the standard t-structure of a derived category the truncation functors are the usual (non-naive) truncation functors.
\end{Ex}

The main result about t-structures in the following theorem that provides a convenient way to identify Abelian subcategories.

\begin{Thm}
    The heart of a t-category is an Abelian category that is stable under extensions (i.e.\ for every distinguished triangle $X → Y → Z → X[1]$ with $X$ and $Z$ in the heart, also $Y$ is in the heart).
\end{Thm}

\begin{Prop}
    The functor $H^0 = τ^{≤0} ∘ τ^{≥0} \cong τ^{≥0} ∘ τ^{≤0}\colon \cat D → \cat D^{\heartsuit}$ is a cohomological functor.
\end{Prop}

We set $H^n = H^0 ∘ [n]$.

\begin{Def}
    Let $F\colon \cat{D₁} → \cat{D₂}$ be a functor of triangulated categories.
    We say that $F$ is \emph{left t-exact} (resp.\ \emph{right t-exact}) if $F(\cat{D₁}^{≥0}) ⊆ \cat{D₂}^{≥0}$ (resp.\ $F(\cat{D₁}^{≥0}) ⊆ \cat{D₂}^{≥0}$).
    It is \emph{t-exact} if it is both left and right t-exact.
\end{Def}

\begin{Prop}
    Let $i\colon \cat{D₁}^\heartsuit → \cat{D₁}$ be the inclusion.
    If $F\colon \cat{D₁} → \cat{D₂}$ is left/right t-exact, then $H^0 ∘ F ∘ i\colon \cat{D₁}^\heartsuit → \cat{D₂}^\heartsuit$ is a left/right exact functor of Abelian categories.
\end{Prop}


\subsection{Operations on sheaves}

$f^*, f_*, f^!, f_!$ in topological and algebraic setting.

\subsection{Perverse constructible sheaves}

With respect to some stratification.

\subsection{Equivariant sheaves}

Definition.
Relation to quotient stacks.

\subsection{Dualizing complexes}

\section{Perverse coherent sheaves}

With the notation introduced earlier, let $G$ act on $X$.
Let $X^\inv$ be the subset of $X^\top$ consisting of the generic points of $G$-invariant subschemes, with the induced topology.

\subsection{Definiton}

\begin{Def}
    A \emph{perversity function} on $X$ is a function $p\colon X^\inv → ℤ$, constant on $G$-orbits.
    The \emph{dual perversity} is defined by $\bar p(x) = -\dim(x) - p(x)$.
    A perversity function $p$ is \emph{(strictly) monotone} if $p(x) < p(x')$ (resp.\ $p(x) \le p(x')$) for $x' ∈ \bar x$.
    It is \emph{(strictly) comonotone} if $\bar p$ is (strictly) monotone. 
\end{Def}

\begin{Def}
    Let $p$ be a perversity function on $X$.
    We define full subcategories $D^{p,\le0}(X) \subset \derivedcoh^-(X)$ and $D^{p,\ge 0}(X) \subset \derivedcoh^+(X)$ by:
    \begin{itemize}
        \item $\sheaf F ∈ D^{p,\le 0}(X)$ if $i_x^*(\sheaf F) ∈ \derived^{\le p(x)}(\catModules{\O_x})$ for all $x ∈ X^\inv$;
        \item $\sheaf F ∈ D^{p,\ge 0}(X)$ if $i_x^!(\sheaf F) ∈ \derived^{\ge p(x)}(\catModules{\O_x})$ for all $x ∈ X^\inv$.
    \end{itemize}
\end{Def}

\begin{Thm}
    $(D^{p,\le0}(X),D^{p,\ge0}(X))$ defines a t-structure on $\derivedcoh(X)$.
\end{Thm}

Before we prove this, let's have a look at an example.

\subsection{Example}

Sheaves on $[\as 2/\SL2]$.

\subsection{Proof}

\begin{Lem}\label{lem:induced_perversity}
    \begin{enumerate}[(a)]
        \item $\dual(D^{p,\le 0}(X)) = D^{\bar p,\ge 0}(X)$
        \item
            Let $i_Z$ be a locally closed $G$-invariant subscheme.
            Then $p$ defines the induced perversity function $p_Z = p ∘ i_Z \colon Z^\inv → ℤ$.
            Then $i_Z^*$ sends $D^{p,\le 0}(X)$ to $D^{p_Z,\le 0}(Z)$ and $i_Z^!$ sends $D^{p,\ge 0}(X)$ to $D^{p_Z,\ge 0}(Z)$.
        \item
            If $Z$ is closed, then ${i_Z}_*$ sends $D^{p_Z,\le 0}(Z)$ to $D^{p,\le 0}(X)$ and $D^{p_Z,\ge 0}(Z)$ to $D^{p,\ge 0}(X)$.
    \end{enumerate}
\end{Lem}

\begin{Lem}\label{lem:Hom(F,G)=0}
    If $\sheaf F ∈ D^{p,\le 0}(X)$ and $\sheaf G ∈ D^{p,\ge 1}(X)$, then $\Hom(\sheaf F,\sheaf G) = 0$.
\end{Lem}


\begin{Thm}
    Suppose that the perversity function $p$ is monotone and comonotone.
    Then $(D^{p,\le0}(X) \cap \derivedcoh^b(X), D^{p,\ge0}(X) \cap \derivedcoh^b(X))$ is a $t$-structure on $\derivedcoh^b(X)$.
\end{Thm}

\begin{proof}
    In view of Lemma \ref{lem:Hom(F,G)=0}, we only need to show that for each object $\sheaf F ∈ \derivedcoh^b(X)$ there exists a distinguished triangle $\sheaf F' → \sheaf F → \sheaf F''$ with $\sheaf F' ∈ D^{p,\le 0}(X)$ and $F'' ∈ D^{p,\ge1}(X)$.
    Symbolically, we need to show that
    \[
    \derivedcoh^b(X) = D^{p,\le 0}(X) * D^{p,\ge1}(X),
    \]
    where for two set of objects $D',D''$ of a triangulated category, $D' * D''$ denotes the set of objects $B$ such that there exists a distinguished triangle $A → B → C$ with $A ∈ D'$ and $B ∈ D''$.
    The octahedral axiom ensures that the operation $*$ is associative \cite[Lemme~1.3.10]{BBD}.

    For a closed $G$-invariant subscheme $i_Z\colon Z \hookrightarrow X$, we set $p_Z = p ∘ i_Z \colon Z^\inv → ℤ$.
    By Noetherian induction, we can assume that the claim is true on every closed invariant subscheme $Z$ with perversity function $p_Z$.

    By Lemma \ref{lem:induced_perversity} we have
    \[
    D^{p,\le0}(X) * {i_Z}_*\left(D^{p_Z,\le 0}(Z)\right) \subseteq D^{p,\le 0}(X) * D^{p,\le 0}(X) = D^{p,\le 0}(X)
    \]
    for every closed $G$-invariant subscheme $Z$ of $X$, and similarly of $D^{p,\ge 0}(X)$.
    Hence,
    \begin{align*}
        D^{p,\le 0}(X) * D^{p,\ge1}(X)
        & \supseteq \bigcup_Z \left(D^{p,\le0}(X) * {i_Z}_*\left(D^{p_Z,\le 0}(Z)\right)\right) * \left({i_Z}_*\left(D^{p_Z,\ge 0}(Z)\right) * D^{p,\ge0}(X)\right) \\
        & = \bigcup_Z D^{p,\le0}(X) * \left( {i_Z}_*\left(D^{p_Z,\le 0}(Z)\right) * {i_Z}_*\left(D^{p_Z,\ge 0}(Z)\right) \right) * D^{p,\ge0}(X) \\
        & = \bigcup_Z D^{p,\le0}(X) * {i_Z}_*\left(\derivedcoh^{b}(Z)\right) * D^{p,\ge0}(X), \\ 
    \end{align*}
    where the last equality follows by induction.
    Thus it suffices to prove that 
    \begin{equation}
        \label{eq:main_theorem:claim}
        \derivedcoh^b(X) = \bigcup_Z D^{p,\le0}(X) * {i_Z}_*\left(\derivedcoh^{b}(Z)\right) * D^{p,\ge0}(X).
    \end{equation}

    For simplicity let us assume that $X$ is irreducible and let $x$ be the generic point of $X$.
    Let $F ∈ \derivedcoh^b(X)$ and set $\sheaf F^- = τ^{\mathrm{stand}}_{\le p(x)}(\sheaf F)$.
    Then $\sheaf F^- ∈ D^{p,\le 0}(X)$ because $p$ is monotone.
    Let $\sheaf F₁$ be the cone of the canonical morphism $\sheaf F^- → \sheaf F$.
    Clearly $i_x^*(\sheaf F₁) ∈ \derived^{>p(x)}(\catModules{\O_x})$.

    Now set 
    \[
    \sheaf F^+ = \dual\left( τ^{\mathrm{stand}}_{<\bar p(x)} \dual(\sheaf F₁) \right).
    \]
    Since $p$ is comonotone, $\sheaf F^+ ∈ D^{p,>0}(X)$.

    \ldots
\end{proof}

\begin{Cor}
    $(D^{p,\le0}(X),D^{p,\ge0}(X))$ defines a t-structure on $\derivedcoh(X)$.
\end{Cor}

\begin{proof}
    We need to show that 
    \[
    \derivedcoh(X) = D^{p,\le 0}(X) * D^{p,\ge1}(X),
    \]
    Let $N$ be sufficiently large, so that we have
    \[
    D^{p,\le 0}(X) \supseteq \derivedcoh^{<-N}(X) 
    \quad\text{and}\quad
    D^{p,\ge 1}(X) \supseteq \derivedcoh^{>N}(X).
    \]
    Then,
    \begin{align*}
         D^{p,\le 0}(X) * D^{p,\ge1}(X)
         & = \derivedcoh^{<-N}(X) *  D^{p,\le 0}(X) *  D^{p,\ge1}(X) * \derivedcoh^{>N}(X) \\
         & \supseteq \derivedcoh^{<-N}(X) *  \derivedcoh^b(X) * \derivedcoh^{>N}(X) \\
         & = \derivedcoh(X).
         \qedhere
     \end{align*}
\end{proof}

\section{Irreducible perverse sheaves}
\subsection{In the constructible case}
\subsection{In the coherent case}
\subsubsection{Example}

\printbibliography

\end{document}
