\documentclass[english]{short-notes}

\usepackage{math-ag}

\addbibresource{global.bib}
%\bibliography{global.bib}

\title{Perverse Sheaves}
\author{Clemens Koppensteiner}

\newcommand\derived{\mathbf D}
\renewcommand\cat{\mathscr}
\let\setset\cover

\begin{document}

\maketitle

\tableofcontents

\section{Some category theory}

We will start by reviewing some general framework from category theory.

\subsection{Triangulated categories}

\begin{Def}
    A \emph{triangulated category} consists of an additive category $\cat D$ together with
    \begin{itemize}
        \item an automorphism (or auto-equivalence) $T\colon \cat D → \cat D$, called the \emph{translation functor}, and
        \item a collection of sequences
            \[ X \xrightarrow{u} Y \xrightarrow{v} Z \xrightarrow{w} TX, \]
            called \emph{distinguished} triangles,
    \end{itemize}
    satisfying the following axioms, where we write $X[1] \coloneq TX$,
        \begin{enumerate}
            \item[(TR~1)]
                Any triangle (i.e.\ sequence $X → Y → Z → X[1]$) isomorphic to a distinguished triangle is distinguished.
                Every morphism $u\colon X → Y$ is contained in a distinguished triangle $X \xrightarrow{u} Y → Z → X[1]$.
                Every triangle of the form $X \xrightarrow{\id} X → 0 → X[1]$ is distinguished.
            \item[(TR~2; rotation)]
                A triangle $X \xrightarrow{u} Y \xrightarrow{v} Z \xrightarrow{w} X[1]$ is distinguished, if and only if $Y \xrightarrow{v} Z \xrightarrow{w} X[1] \xrightarrow{-u[1]} Y[1]$ is distinguished.
            \item[(TR~3; morphisms)] 
                If $X \xrightarrow{u} Y \xrightarrow{v} Z \xrightarrow{w} X[1]$ and
                $X' \xrightarrow{u'} Y' \xrightarrow{v'} Z' \xrightarrow{w'} X'[1]$
                are distinguished, then for each morphism $(f,g)\colon u → u'$ there exists a morphism $h\colon Z → Z'$ such that $(f,g,h)$ is a morphism of triangles (where a morphism of triangles is defined as the obvious commutative diagram).
            \item[(TR~4; octahedral axiom)]
                If $X \xrightarrow{u} Y \xrightarrow{u} Z' → X[1]$,
                $Y \xrightarrow{v} Z \xrightarrow{} X' \xrightarrow{j} Y[1]$, and
                $X \xrightarrow{w} Z \xrightarrow{} Y' \xrightarrow{} X[1]$
                are distinguished triangles such that $w = v∘u$, then there exist morphisms $f\colon Z' → Y'$ and $g\colon Y' → X'$ such that
                \begin{enumerate}[1)]
                    \item $(\id[X],v,f)$ is a morphism of triangles;
                    \item $(u, \id[Z],g)$ is a morphism of triangles;
                    \item $Z' \xrightarrow{f} Y' \xrightarrow{g} X' \xrightarrow{i[1]} Z'[1]$ is a distinguished triangle.
                \end{enumerate}
    \end{enumerate}
    A \emph{functor of triangulated categories} is a functor of the underlying categories which commutes with the translation functors and sends distinguished triangles to distinguished triangles.
\end{Def}

The notion of triangulated categories was first formalized in \cite{Verdier:1977:CategoriesDerivees}.
Some explanation of the octahedral axiom can be found in \cite[Section~1.1]{BeilinsonBernsteinDeligne:1982:FaisceauxPervers}.
Of course, the theory can also be found in many textbooks, e.g. \cite{KashiwaraSchapira:2006:CategoriesAndSheaves} or \cite{Weibel:1994:IntroToHomologicalAlgebra}.

\begin{Rem}
    As David noted, triangulated categories are not the \enquote{right} notion.
    A better notion are stable $\infty$-categories, where the analogue to distinguished triangles is not additional data added to the category, but defined more intrinsically.
    See \cite[Chapter 1]{Lurie:2011-draft:HigherAlgebra} (that chapter also contains some discussion of t-structures in both the classical and $\infty$-category setting).
\end{Rem}

\begin{Ex}
    Let $\cat A$ be an additive category and $\mathbf K\cat A$ the category of complexes of objects of $\cat A$ and $\Hom_{\mathbf K\cat A}(K,L)$ the set of homotopy classes of morphisms of complexes from $K$ to $L$.
    Let $TK = K[1]$ be defined by $(K[1])^n = K^{n+1}$ (with a sign change in the differential).
    In $\mathbf K\cat A$ the distinguished triangles are gives by $K \xrightarrow{u} L → \operatorname{cone}(u) → K[1]$, giving $\mathbf K\cat A$ the structure of a triangulated category.
    Here we need to work in the homotopy category of complexes, as otherwise the triangle $K \xrightarrow{\id} K → 0 → X[1]$ wouldn't be distinguished (the mapping cone of $\id$ is only quasi-isomorphic to 0).
    Also note that if $u\colon K → L$ is injective, then $\operatorname{cone}(u) \cong L/K$, so that every short exact sequence of complexes is a distinguished triangle.

    If $\cat A$ is Abelian, this structure descends to the derived category $\derived \cat A$.
\end{Ex}

\begin{Ex}
    The stable homotopy category (of spectra) is a triangulated category (where the distinguished triangles are given by mapping cones of spectra).
\end{Ex}

\begin{Def}
    Let $\cat D$ be a triangulated category and $\cat A$ an Abelian category.
    A functor $F\colon \cat D → \cat A$ is called \emph{cohomological} if it maps each distinguished triangle $X → Y → Z → X[1]$ to an exact sequence $F(X) → F(Y) → F(Z)$.
\end{Def}

\begin{Ex}
    For any $X ∈ \cat D$ the functors $\Hom_{\cat D}(X,\cdot)$ and $\Hom_{\cat D}(\cdot, X)$  are cohomological (the second one being defined on the opposite triangulated category (with translation functor the inverse of $T$ and reading distinguished triangles the other way)).
    Hence, if $F\colon \cat D → \cat A$ is cohomological, it induces a long exact sequence
    \[
    \dotsc → FX → FY → FZ → FTX → FTY → \dotsc
    \]
    for each distinguished triangle $X → Y → Z → X[1]$ (using TR~2 repeatedly).
\end{Ex}

\subsection{t-categories}

\begin{Def}
    A \emph{t-category} is a triangulated category $\cat D$ together with a pair $(\cat D^{≤0},\cat D^{≥0})$ of full subcategories, called a \emph{t-structure}, such that, writing $\cat D^{≤n} = \cat D^{≤0}[n]$ and $\cat D^{≥n} = \cat D^{≥0}[n]$, the following conditions hold:
    \begin{enumerate}
        \item if $X ∈ \cat D^{≤0}$ and $Y ∈ \cat D^{≥1}$, then $\Hom_{\cat D}(X,Y) = 0$;
        \item $\cat D^{≤0} ⊆ \cat D^{≤1}$ and $\cat D^{≥0} ⊇ \cat D^{≥1}$;
        \item for each $X ∈ \cat D$ there exists a distinguished triangle $A → X → B → A[1]$ such that $A ∈ \cat D^{≤0}$ and $B ∈ \cat D^{≥1}$.
    \end{enumerate}

    The full subcategory $\cat D^{≤0} ∩ \cat D^{≥0}$ is called the \emph{heart} of the t-structure, sometimes denoted $\cat D^\heartsuit$.
\end{Def}

The notion of t-categories was first introduced in \cite{BeilinsonBernsteinDeligne:1982:FaisceauxPervers}.
Modern references include almost any book that describes perverse sheaves, e.g. \cite{HottaTakeuchiTanisaki:2008:DModulesPerverseSheavesRepresentationTheory, PetersSteenbrink:2008:MixedHodgeStructures, KashiwaraSchapira:1994:SheavesOnManifolds}.

\begin{Ex}
    If $\cat A$ is an Abelian category, then its derived category $\derived \cat A$ admits a standard t-structure given by
    \begin{align*}
        \derived \cat A^{≤0} &= \{ F ∈ \derived\cat A : H^j(F) = 0 \text{ for all } j > 0 \}, \\
        \derived \cat A^{≥0} &= \{ F ∈ \derived\cat A : H^j(F) = 0 \text{ for all } j < 0 \}.
        \qedhere
    \end{align*}
\end{Ex}

\begin{Prop}
    The inclusions $\cat D^{≤n} → \cat D$ (resp.~$\cat D^{≥n} → \cat D$) admit right (resp.~left) adjoint functors $τ^{≤n}\colon \cat D → \cat D^{≤n}$ (resp.\ $τ^{≥n}\colon \cat D → \cat D^{≥n}$), called the \emph{truncation functors}.
\end{Prop}

These functors $τ^{≤0}$ and $τ^{≥1}$ are given by the third property of the definition of a t-structure. 
The other truncation functors are just translates.

\begin{Ex}
    For the standard t-structure of a derived category the truncation functors are the usual (non-naive) truncation functors.
\end{Ex}

The main result about t-structures in the following theorem that provides a convenient way to identify Abelian subcategories.

\begin{Thm}
    The heart of a t-category is an Abelian category that is stable under extensions (i.e.\ for every distinguished triangle $X → Y → Z → X[1]$ with $X$ and $Z$ in the heart, also $Y$ is in the heart).
\end{Thm}

\begin{Prop}
    The functor $H^0 = τ^{≤0} ∘ τ^{≥0} \cong τ^{≥0} ∘ τ^{≤0}\colon \cat D → \cat D^{\heartsuit}$ is a cohomological functor.
\end{Prop}

We set $H^n = H^0 ∘ [n]$.

\begin{Def}
    Let $F\colon \cat{D₁} → \cat{D₂}$ be a functor of triangulated categories.
    We say that $F$ is \emph{left t-exact} (resp.\ \emph{right t-exact}) if $F(\cat{D₁}^{≥0}) ⊆ \cat{D₂}^{≥0}$ (resp.\ $F(\cat{D₁}^{≥0}) ⊆ \cat{D₂}^{≥0}$).
    It is \emph{t-exact} if it is both left and right t-exact.
\end{Def}

\begin{Prop}
    Let $i\colon \cat{D₁}^\heartsuit → \cat{D₁}$ be the inclusion.
    If $F\colon \cat{D₁} → \cat{D₂}$ is left/right t-exact, then $H^0 ∘ F ∘ i\colon \cat{D₁}^\heartsuit → \cat{D₂}^\heartsuit$ is a left/right exact functor of Abelian categories.
\end{Prop}

\section{Perverse sheaves}

\subsection{\ldots on a stratified space}

For this subsection, let $X$ be a topological space and $\setset S$ a finite stratification with each stratum locally closed.
We assume that the closure of each stratum is a union of strata.
Let $\O$ be a sheaf of rings on $X$.

\begin{Def}[{\cite[Définition~2.1.2]{BeilinsonBernsteinDeligne:1982:FaisceauxPervers}}]
    A function $p\colon \setset S → ℤ$ is called a \emph{perversity}.
    Define the following full subcategories of $\derived(X,\O)$
    \begin{align*}
        \perv\derived^{≤0}(X,\O) & = \{ K ∈ \derived(X,\O) : H^ni_S^*K = 0 \text{ for each } S ∈ \setset S \text{ and } n > p(S) \}, \\
        \perv\derived^{≥0}(X,\O) & = \{ K ∈ \derived^+(X,\O) : H^ni_S^!K = 0 \text{ for each } S ∈ \setset S \text{ and } n < p(S) \},
    \end{align*}
    where $i_S \colon S \hookrightarrow X$ is the inclusion.
\end{Def}

Recall that $i_S^*$ is the derived functor of $i_S^*(\sheaf G) = f^{-1}\sheaf G \otimes_{f^{-1}\O_S} \O_X$ and $i_S^!(\sheaf G) =  f^*\sheaf H$, where $\sheaf H(U) = \{ s ∈ G(U) : \supp s ⊆ U\}$ (for $S$ locally closed).

\begin{Thm}
    $(\perv\derived^{≤0}(X,\O),  \perv\derived^{≥0}(X,\O))$ is a t-structure on $\derived (X,\O)$ and induces one on $\derived^*(X,\O)$ for $*={+},{-},{b}$.
\end{Thm}

The objects in the heart of this t-structure are called \emph{perverse sheaves} on $X$ (with respect to $\setset S$, $p$ and $\O$). 

\subsection{\ldots on a complex variety}

This definition is more general than what we need for our purposes.
From now on let $X$ be a complex algebraic variety or a complex manifold.
Instead of a general sheaf of rings $\O$, we will only consider the constant sheaf $ℂ_X$ on $X$.
Let us also agree that all strata are smooth and equidimensional complex subvarieties/submanifolds and that the value of the perversity $p$ only depends on the dimension of $S$.

Let $\derived_{c-\setset S}(X)$ be the full subcategory of $\derived(\catModules{ℂ_X})$ consisting of objects with constructible cohomology with respect to $\setset S$ (a sheaf is constructible with respect to $\setset S$ if the restrictions to each stratum is a locally constant sheaf of $ℂ$-modules).
Then the construction above gives an Abelian category of perverse sheaves on $X$ as the heart of the perverse t-structure on $X$ with respect to some perversity.

\begin{Exercise}
    Let $X = ℂ$, $\setset S = \{\{0\},ℂ^\units\}$ and let $p(\{0\}) = 0$, $p(ℂ^\units) = -1$.
    Describe the perverse sheaves with respect to these data.
    See \cite{GelfandMacPhersonVilonen:1996:PerverseSheavesAndQuivers} for a generalization of this example.

    As a simplification, describe the $ℂ^\units$-invariant perverse sheaves on $ℂ$.
\end{Exercise}

We would like to get a notion of perverse sheaves that is independent of the stratification.
For that we notice that a refinement of the stratification induces compatible t-structures on the categories $\derived_{c-\setset S}(X)$.
So we can try to take a limit over all stratifications.

For this to work, we need to restrict the class of allowed perversities.
Let $p\colon ℤ → ℤ$ be a perversity.
The \emph{dual perversity} is defined by $p^*(n) = -n - p(n)$.
We need to assume that both $p$ and $p^*$ are decreasing, i.e.\ that for $m ≤ n$ we have
\[ 0 ≤ p(m) - p(n) ≤ n-m. \]
Then for a stratum $S$ we set
\[
p(S) = p(\dim_{\mathrm{top}} S) = p(2\dim_{\mathrm{alg}} S).
\]

Let $\derived_c(X)$ be the full subcategory of $\derived(\catModules{ℂ_X})$ consisting of objects with constructible cohomology (i.e.\ constructible in the above sense for some (nice) stratification).
We will restrict our attention to $\derived^b_c(X)$.

\begin{Prop}
    Let $\perv\derived^{≤0}(X)$ be the full subcategory of $\derived_c^b(X)$ consisting of objects $F$ for which $H^j(i_S^*F) = 0$ for every locally closed complex analytic subset $S$ and any $j > p(S)$.
    
    Let $\perv\derived^{≥0}(X)$ be the full subcategory of $\derived_c^b(X)$ consisting of objects $F$ for which $H^j(i_S^!F) = 0$ for every locally closed complex analytic subset $S$ and any $j < p(S)$.

    Then $(\perv\derived^{≤0}(X),\perv\derived^{≥0}(X))$ is a t-structure on $\derived_c^b(X)$.
\end{Prop}

Let $\mathbb D_X\colon \oppcat{\derived_c^b(X)} \isoto \derived_c^b(X)$ be the Verdier dual (see for example \cite[Section 13.1]{PetersSteenbrink:2008:MixedHodgeStructures} or \cite[Section 4.5]{HottaTakeuchiTanisaki:2008:DModulesPerverseSheavesRepresentationTheory}).

\begin{Prop}
    The Verdier dual exchanges $\perv\derived^{≤0}(X)$ and $\perv[p^*]\derived^{≥0}(X)$.
\end{Prop}

In the complex case (which we are discussing) the most important perversity is the \emph{middle perversity} $p(2k) = -k$ (the value on odd integers in irrelevant).
This is the only self-dual perversity.
In particular $\mathbb D_X\oppcat{\perv\derived^{≤0}(X)} = \perv\derived^{≥0}(X)$.
From now on, we will only consider this perversity.

\begin{Prop}
    Let $F ∈ \derived_c^b(X)$.
    The following are equivalent:
    \begin{enumerate}
        \item $F ∈ \perv\derived^{≤0}(X)$;
        \item $\dim\supp H^j(F) ≤ -j$ for all $j$.
    \end{enumerate}
    The following are equivalent:
    \begin{enumerate}
        \item $F ∈ \perv\derived^{≤0}(X)$;
        \item $\dim\supp H^j(\mathbb D_X F) ≤ -j$ for all $j$.
    \end{enumerate}
    In particular, $\mathbb D_X$ is t-exact and induces a functor $\oppcat{\operatorname{Perv}(X)} \isoto \operatorname{Perv}(X)$.
\end{Prop}

\begin{Ex}
    Let $X$ be smooth of dimension $d$ and $L$ a local system on $X$.
    Then $L[d]$ is a perverse sheaf on $X$.

    As, using that $ω_X \cong ℂ[2d]$,
    \[
    \mathbb D_X(L[d]) \cong
    R\sheaf{Hom}_{ℂ}(L[d], ℂ[2d]) \cong
    L^*[d].
    \]
    More generally this is true for $X$ a pure-dimensional algebraic variety that is locally a complete intersection.
\end{Ex}

In general the (exceptional) pull-back and push-forward functors are not t-exact with respect to the perverse t-structure (see however, e.g., \cite[Corollary~8.1.44]{HottaTakeuchiTanisaki:2008:DModulesPerverseSheavesRepresentationTheory} for cases where they are exact).
As explained above, to any functor $F\colon\derived_c^b(X) → \derived_c^b(Y)$ we can associate a functor $\perv F$ between the hearts of the perverse t-structures, i.e.~between the perverse sheaves on the two categories.

Let $j\colon U → X$ be the inclusion of a dense open subset.
We have a natural morphism $j_! → j_*$ of functors\footnote{Recall that $f_!\sheaf F(V) = \{ s ∈ F(f^{-1}V) : \res{f}{\supp s}\colon \supp s → U\text{ is proper}\}$.}.
The \emph{intermediate extension} of a perverse sheaf $F$ on $U$ is defined to be
\[
\perv j_{!*} F = \im(\perv j_! F → \perv j_*F).
\]
The use of this functor is to recursively define the \emph{intersection complex}, which then calculates intersection cohomology (see \cite[Section~8.2]{HottaTakeuchiTanisaki:2008:DModulesPerverseSheavesRepresentationTheory}.
It can also be used to describe the simple objects in the category of perverse sheaves \cite[Corollary~8.2.10]{HottaTakeuchiTanisaki:2008:DModulesPerverseSheavesRepresentationTheory}, \cite[Lemma~13.26]{PetersSteenbrink:2008:MixedHodgeStructures}.

\subsection{Some notes on the literature}

Perverse sheaves where first described in \cite{BeilinsonBernsteinDeligne:1982:FaisceauxPervers} and most fundamental results can be found there.
If one is only interested in complex geometry (and the middle perversity), then \cite{HottaTakeuchiTanisaki:2008:DModulesPerverseSheavesRepresentationTheory} and \cite{PetersSteenbrink:2008:MixedHodgeStructures} contain nice introductions.
For perverse sheaves on real and complex manifolds \cite{KashiwaraSchapira:1994:SheavesOnManifolds} is a detailed reference.
For complex manifolds, \cite{KashiwaraSchapira:1994:SheavesOnManifolds} also contains a microlocal description of perverse sheaves.

\printbibliography

\end{document}
