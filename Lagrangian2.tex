\documentclass[english]{short-notes}

\title{Perverse sheaves and half-dimensional subvarieties}
\author{Clemens Koppensteiner}

\usepackage{math-ag}

\addbibresource{global.bib}
%\bibliography{global.bib}

\newcommand\dualize{\mathbb D}

\begin{document}

\maketitle
\tableofcontents

\section{Setup and notation}

Let $X$ be a complex\footnote{This assumption is probably not necessary.} algebraic variety.
We write $\O = \O_X$ for the structure sheaf of $X$.
Let $G$ be a complex algebraic group acting on $X$.
Except in Section~\ref{sec:Kashiwara}, we will assume that the $G$-action has finitely many orbits, and all orbits are even-dimensional.
If $U_{2n}$ is the union of all $2n$-dimensional orbits, then we assume that $\bigcup_{n < k} U_{2n}$ is always closed\footnote{Is this always true? Is there a simple condition for this to be true?}.
We write $X^{\mathrm{top}}$ for the subset of the topological space of $X$ consisting of generic points of $G$-invariant subschemes.

We write $D(X)$, $D_{qc}(X)$ and $D_c(X)$ for the derived category of $\O_X$-modules and its full subcategories consisting of complexes with quasi-coherent and coherent cohomology sheaves respectively.
The corresponding categories for $G$-equivariant sheaves (i.e.\ the categories for the quotient stack $[X/G]$) will be denoted $D(X^G)$, $D_{qc}(X^G)$ and $D_c(X^G)$.
As usual, $D^b(X)$ (etc.) will be the full subcategory of $D(X)$ consisting of complexes with cohomology in only finitely many degrees.
All functors will be derived, though we will usually drop the $R$ or $L$.

Let $Z$ be a closed subset of $X$, and $\sheaf I$ an ideal sheaf of $\O_X$ such that $Z = \supp(\rquot{\O_X}{\sheaf I})$.
For an $\O_X$-module $\sheaf F$ we let $Γ_Z\sheaf F$ be subsheaf of $\sheaf F$ of sections with support in $Z$ \cite[Varition~3 in IV.1]{Hartshorne:1966:ResiduesAndDuality}.
%If $\sheaf F$ is quasi-coherent, then by \cite[Theorem~V.4.1]{Hartshorne:1966:ResiduesAndDuality} we have
%\[
%Γ_Z\sheaf F = \varinjlim_n \sheafHom_{\O_X}(\rquot{\O_X}{\sheaf I^n}, \sheaf F).
%\]
By abuse of notation, we will usually simply write $Γ_Z$ for the right-derived functor $RΓ_Z\colon D(X) → D(X)$.

Let $x$ be a (not necessarily closed) point of $x$ and $\sheaf F ∈ D^b(X)$.
Then $\mathbf ι_x^*\sheaf F = \sheaf F_x ∈ D^b(\catModules{\O_x})$ will denote the (derived) functor of talking stalks.
We further set $\mathbf ι_x^!\sheaf F = \mathbf ι_x^*Γ_{\overline x}$.

We assume that $X$ has a $G$-equivariant dualizing complex $\sheaf R$ (see \cite[Definition~1]{Bezrukavnikov:arXiv:PerverseCoherentSheaves}) which we assume to be normalized.
For $\sheaf F ∈ D(X)$ (or $D(X^G)$) we write $\dualize \sheaf F = \sheafHom_{\O_X}(\sheaf F,\sheaf R)$ for its dual.
Then we have
\begin{itemize}
    \item $\dualize\colon D^b_{c}(X) → D^b_c(X)$ (resp.~$\dualize\colon D^b(X^G) → D^b(X^G)$).
    \item For $\sheaf F ∈ D^b_c(X)$ (or $D^b_c(X^G)$), the natural morphism $\sheaf F → \dualize\dualize F$ is an isomorphism.
    \item For $x ∈ X$, $\mathbf ι_x^!(\sheaf R)$ is concentrated in $-\dim x$ (where $\dim x = \dim\overline x$), see \cite[Section~3.1]{Bezrukavnikov:arXiv:PerverseCoherentSheaves}.
\end{itemize}

Recall that Bezrukavnikov defined a t-structure on $D_c^b(X^G)$ in the following way.
Let $p\colon X^{\mathrm{top}} → ℤ$ be a function, called $\emph{perversity}$.
We assume that $p$ is monotone (i.e.\ if $x' ∈ \overline x$, then $p(x') ≥ p(x)$) and comonotone (i.e.\ $\overline p(x) = -\dim x - p(x)$ is monotone).
Then the two full subcategories of $D_c^b(X^G)$ by 
\begin{align*}
    \perv[p] D^{≤0}(X) & = 
    \{ \sheaf F ∈ D_c^b(X^G) : \mathbf ι_x^*\sheaf F ∈ D^{≤p(x)}(\catModules{\O_x}) \text{ for all $x ∈ X^{\mathrm{top}}$}\}, \\
    \perv[p] D^{≥0}(X) & = 
    \{ \sheaf F ∈ D_c^b(X^G) : \mathbf ι_x^!\sheaf F ∈ D^{≥p(x)}(\catModules{\O_x}) \text{ for all $x ∈ X^{\mathrm{top}}$}\}
\end{align*}
defines a t-structure on $D_c^b(X^G)$.
The heart of this t-structure is called the category of perverse sheaves (with respect to the perversity $p$).

Our goal is to give a new description of the category of perverse sheaves with respect to the middle perversity $m$ defined by $m(x) = -\frac12 \dim x$.
Note that $m = \overline m$ and hence $\dualize \perv[m] D^{≤0}(X) = \perv[m] D^{≥0}(X)$ \cite[Lemma~5(a)]{Bezrukavnikov:arXiv:PerverseCoherentSheaves}.

\section{Kashiwara's definition}
\label{sec:Kashiwara}%

For this section we do not assume that the $G$-action has finitely many even-dimensional orbits.
In particular the action may be trivial (this case is used in the proof of Lemma~\ref{lem:supportAndLocalCohomology+}.

In \cite{Kashiwara:2004:tStructureOnHolonomicDModuleCoherentOModules}, Kashiwara also gave a definition of a perverse t-structure on $D^b_{coh}(X)$.
The two definitions match (up to a shift by $\dim X$), but there doesn't seem to be a reference for this is the literature.
So we will prove it in the following proposition:

\begin{Prop}
    \label{prop:equivDeligneKashiwara}%
    Let $p$ be a monotone and comonotone perversity on $X^{\mathrm{top}}$ such that $p(x)$ depends only on $\dim x$.
    By abuse of notation, let $p\colon ℤ → ℤ$ be the induced function so that $p(x) = p(\dim x)$.
    Then,
    \begin{align*}
        \perv D^{≤0}(X) & = 
        \{ \sheaf F ∈ D_c^b(X^G) : p(\dim \supp H^{k}(\sheaf F)) ≥ k \text{ for all $k$}\}; \\
        \perv D^{≥0}(X) & = 
        \{ \sheaf F ∈ D_c^b(X^G) : Γ_{\overline x}\sheaf F ∈ D^{≥p(x)}(X) \text{ for all $x ∈ X^{\mathrm{top}}$}\}.
    \end{align*}
\end{Prop}

\begin{proof}
    First assume that $\sheaf F ∈ \perv D^{≤0}(X)$ and assume that there exists an integer $k$ such that $p(\dim \supp H^{k}(\sheaf F)) < k$.
    Let $x$ be the generic point of an irreducible component of $\supp H^{k}(\sheaf F)$.
    Then $\dim x = \supp H^{k}(\sheaf F)$ and $H^k(\mathbf ι_x^* \sheaf) \ne 0$.
    But $\mathbf ι_x^*\sheaf F ∈ D^{≤p(x)}(\catModules{\O_x})$, a contradiction.

    Conversely assume that $p(\dim \supp H^{k}(\sheaf F)) ≥ k$ for all $k$ and let $x ∈ X^{\mathrm{top}}$.
    We have $H^k(\mathbf ι_x^*\sheaf F) \ne 0$ if and only if $\overline x ⊆ \supp H^{k}(\sheaf F)$ (and hence $\dim x ≤ \dim \supp H^{k}(\sheaf F)$).
    Thus monotonicity of the perversity implies that $\sheaf F ∈ \perv D^{≤0}(X)$.

    Now assume that $\sheaf F ∈ \perv D^{≥0}(X)$.
    We proceed by induction on $\dim x$.
    If $\dim x = 0$, then $\mathbf ι_x^!\sheaf F = Γ(X,Γ_{\overline x})$ and thus $Γ_{\overline x}\sheaf F ∈ D^{≥p(x)}(X)$ by assumption.
    For the induction step there are two cases:
    If $\supp Γ_{\overline x} = \overline x$, then $H^k(\mathbf ι_x^!\sheaf F) = H^k(\mathbf ι_x^*Γ_{\overline x} \sheaf F)$ vanishes if and only if $H^k(Γ_{\overline x})$ vanishes and we are done.
    If $\supp Γ_{\overline x} ⊆ \overline x$ then the implication follows by induction and monotonicity of the perversity.

    The last implication follows directly from the relation $\mathbf ι_x^!\sheaf F = \mathbf ι_x^*Γ_{\overline x} \sheaf F$.
\end{proof}

We need to show that several results from \cite{Kashiwara:2004:tStructureOnHolonomicDModuleCoherentOModules} hold in the non-regular case, with $\sheafHom(-,\O_X)$ replaced by $\dualize_X(-)$.

Combining the proposition with \cite[Lemma~5]{Bezrukavnikov:arXiv:PerverseCoherentSheaves} we obtain the following corollary as an immediate consequence.

\begin{Cor}[{\cite[Proposition~4.3]{Kashiwara:2004:tStructureOnHolonomicDModuleCoherentOModules}}]
    \label{cor:dualStandard}
    The dualizing functor $\dualize_X$ sends the standard t-structure on $D_c^b(X^G)$ to the one associated to the perversity $p(x) = -\dim x$.
\end{Cor}


\section{Measuring subvarieties}

\begin{Def}
    A \emph{measuring subvariety} of $X$ is an irreducible subvariety $Z$ of $X$ such that $\dim(U ∩ Z) = \frac12 \dim U$ for each $G$-orbit $U$ with $U ∩ Z \ne \emptyset$.
\end{Def}

TODO: Existence of measuring subvarieties.% (and transversal pairs of measuring pairs).

Our main theorem will easily follow from the next two lemmas.

\begin{Lem}
    \label{lem:supportAndLocalCohomology-}%
    Let $\sheaf F ∈ \catCoh{X^G}$ be a $G$-equivariant coherent sheaf on $X$ and let $n$ be an integer.
    Then the following are equivalent:
    \begin{enumerate}
        \item $\dim \supp \sheaf F ≤ 2n$;
        \item $H^\ell(Γ_Z\sheaf F) = 0$ for all $\ell ≥ n+1$ and all measuring subvarieties $Z$.
    \end{enumerate}
\end{Lem}

\begin{proof}
    First we will show that (i) implies to (ii).
    We will do so by induction on $n$.
    The statement is trivial for $n = 0$.

    Since $\supp \sheaf F$ is always a union of orbits, we can restrict to the support and assume that $\supp \sheaf F = X$.
    If $\dim X < 2n$, we are done by induction, so let us assume that $\dim X = 2n$.
    Let $U$ be the union of all $2n$-dimensional orbits and let $Y$ be its complement.
    By assumption $U$ is open and $Y$ closed.
    Give $Y$ the reduced subscheme structure.
    From adjunction we get a surjective morphism $φ\colon \sheaf F → {ι_Y}_* {ι_Y}^* \sheaf F$ (here ${ι_Y}_*$ and ${ι_Y}^*$ are the underived $\O$-module operations, sending coherent sheaves to coherent sheaves).
    Let $\sheaf G$ be the kernel of this map, so that we get a short exact sequence
    \[
    0 → \sheaf G → \sheaf F → {ι_Y}_* {ι_Y}^* \sheaf F → 0
    \]
    in $\catCoh{X^G}$.
    By induction, $H^\ell(Γ_Z{ι_Y}_* {ι_Y}^*\sheaf F) = 0$ for all $\ell > n-1$.
    So we only need to show that $H^\ell(Γ_Z\sheaf G) = 0$ for all $\ell > n$.
    Since $\sheaf G$ is supported on the open set $U$, we can restrict to $U$ and compute $Γ_{Z∩U}\res{\sheaf G}U$.
    As $\sheaf G$ is a coherent $G$-equivariant sheaf on the smooth variety $U$, it is locally free\footnote{Reference?}.
    Thus\footnote{Add reference or lemma.} $H^\ell(Γ_{Z∩U}\res{\sheaf G}U) = 0$ for $\ell > n = \codim_ZU$, as required.

    Now assume conversely that $H^\ell(Γ_Z\sheaf F) = 0$ for all $\ell ≥ n+1$ and all measuring subvarieties $Z$.
    We have to show that $\dim \supp \sheaf F ≤ 2n$.
    Assume that this is not true, so that $d = \dim Z = \frac12 \dim \supp \sheaf F ≥ n+1$.
    We can assume that $X$ is affine and write $X = \Spec A$.\footnote{There should be a reference for local cohomology of schemes where this assumption is not necessary.} 
    By flat base change,
    \[
    Γ(X,H^d(Γ_Z \sheaf F))_{\ideal p} = 
    \left(H_{\ideal p}^d(Γ(X,\sheaf F))\right)_{\ideal p} =
    H_{\ideal p_{\ideal p}}^d(Γ(X,\sheaf F)_{\ideal p})
    \]
    By the Grothendieck non-vanishing theorem \cite[Theorem~6.1.4]{BrodmannSharp:1998:LocalCohomology} this is non-zero\footnote{Do I need to argue that $Γ(X,\sheaf F)_{\ideal p} \ne 0$? It must be true for at least one $Z$, so just pick one that lies in the open orbit?}, and hence $Γ(X,H^d(Γ_Z \sheaf F)) \ne 0$, contradicting the assumption.
\end{proof}

\begin{Lem}[{\cite[Proposition~5.2]{Kashiwara:2004:tStructureOnHolonomicDModuleCoherentOModules}}]
    \label{lem:supportAndLocalCohomology+}%
    Let $\sheaf F ∈ D_c^b(X)$, $Z$ a closed subset of $X$ and $n$ an integer.
    Then $Γ_Z\sheaf F ∈ D_{qc}^{≥n}(X)$ if and only if $-\dim(Z∩\supp(H^k(\dualize \sheaf F))) ≥ k + n$ for all $k$.
\end{Lem}

\begin{proof}
    The proof of \cite[Proposition~5.2]{Kashiwara:2004:tStructureOnHolonomicDModuleCoherentOModules} works for singular schemes as well --- just substitute $\dualize M$ for $M^*$ and Corollary \ref{cor:dualStandard} for \cite[Proposition~4.3]{Kashiwara:2004:tStructureOnHolonomicDModuleCoherentOModules}.
\end{proof}

\begin{Thm}
    Let $\sheaf F ∈ D_c^b(X^G)$.
    Then $\sheaf F$ is perverse with respect to the middle perversity if and only if\/ $Γ_Z\sheaf F$ is cohomologically concentrated in degree $0$ for each measuring subvariety $Z$.
    More precisely,
    \begin{enumerate}
        \item $\perv[m] D^{≤0}(X) = \{ \sheaf F ∈ D_c^b(X^G) : Γ_Z\sheaf F ∈ D^{≤0}(X) \text{ for all measuring subvarieties $Z$}\}$;
        \item $\perv[m] D^{≥0}(X) = \{ \sheaf F ∈ D_c^b(X^G) : Γ_Z\sheaf F ∈ D^{≥0}(X) \text{ for all measuring subvarieties $Z$}\}$.
    \end{enumerate}
\end{Thm}

\begin{proof}
    The first statement follows immediately from (i) and (ii), so we will prove those.
\begin{enumerate}
\item 
    We will use the description of $\perv[m] D^{≤0}(X)$ given by Proposition~\ref{prop:equivDeligneKashiwara}, i.e.
    \[
    \perv[m] D^{≤0}(X) = \{ \sheaf F ∈ D_c^b(X^G) : \dim\left( \supp H^{-n}(\sheaf F) \right) ≤ 2n \text{ for all $n$}\}.
    \]
    We will use induction on the largest $k$ such that $H^k(\sheaf F) \ne 0$ to show that $\sheaf F ∈ \perv[m]D^{≤0}$ if and only if $Γ_Z\sheaf F ∈ D^{≤0}(X)$ for all measuring subvarieties $Z$.

    The equivalence is trivial for $k \ll 0$.
    For the induction step note that there is a distinguished triangle
    \[
    τ_{≤ k-1} \sheaf F → \sheaf F → H^k(\sheaf F)[-k] \xrightarrow{+1}.
    \]
    Applying the functor $Γ_Z$ and taking cohomology we obtain 
    \begin{multline*}
        \cdots →
        H¹(Γ_Z(τ_{≤ k-1} \sheaf F)) →
        H¹(Γ_Z\sheaf F) →
        H^{k+1}(Γ_Z(H^k(\sheaf F))) → \\
        H²(Γ_Z(τ_{≤ k-1} \sheaf F)) →
        H²(Γ_Z\sheaf F) →
        H^{k+2}(Γ_Z(H^k(\sheaf F))) →
        \cdots.
    \end{multline*}
    By induction, $H^\ell(Γ_Z(τ_{≤ k-1} \sheaf F))$ vanishes for $\ell ≥ 1$ so that $H^\ell(Γ_Z\sheaf F) \cong H^{k+\ell}(Γ_Z(H^k(\sheaf F)))$ for $\ell ≥ 1$.
    Thus the statement follows from Lemma~\ref{lem:supportAndLocalCohomology-}.
\item 
    By Proposition~\ref{prop:equivDeligneKashiwara} and Lemma~\ref{lem:supportAndLocalCohomology+}, $\sheaf F ∈ \perv[m] D^{≥0}$ if and only if
    \begin{equation}
        \label{eq:main:+supp1}%
        \dim\left( \overline x ∩ \supp\left( H^k(\dualize F) \right) \right) ≤ \frac12\dim x - k \qquad \text{ for all $x ∈ X^{\mathrm{top}}$ and all $k$}.
    \end{equation}
    Using Lemma~\ref{lem:supportAndLocalCohomology+} for $Γ_Z\sheaf F ∈ D^{≥0}(X)$, we see that we have to show the equivalence of \eqref{eq:main:+supp1} with
    \begin{equation}
        \label{eq:main:+supp2}%
        \dim\left( Z ∩ \overline x ∩ \supp\left( H^k(\dualize F) \right) \right) ≤ - k \qquad \text{ for all $x ∈ X^{\mathrm{top}}$, all $k$ and all measuring $Z$}.
    \end{equation}
    Fix a $k$.
    Let us first show the implication from \eqref{eq:main:+supp1} to \eqref{eq:main:+supp2}.
    Since $H^k(\dualize \sheaf F)$ is $G$-equivariant, it suffices to show \eqref{eq:main:+supp2} assuming that $\overline x ⊆ \supp H^k(\dualize F)$.
    But then $\overline x ∩ \supp\left( H^k(\dualize F) \right) = \overline x$ and $Z ∩ \overline x ∩ \supp\left( H^k(\dualize F) \right) = Z ∩ \overline x$ and $\dim(Z ∩ \overline x) = \frac 12 \dim x$, yielding the implication.

    Conversely, assume that \eqref{eq:main:+supp2} holds.
    If $\overline x ⊆ \supp H^k(\dualize F)$, then the exact same argument as before shows that \eqref{eq:main:+supp1} holds for $x$.
    Otherwise $\overline x ∩ \supp\left( H^k(\dualize F) \right) = \overline y$ for some $y ∈ X^{\mathrm{top}}$ with $\dim y < \dim x$.
    Then \eqref{eq:main:+supp1} holds for $y$ and hence
    \[
    \dim\left( \overline x ∩ \supp\left( H^k(\dualize F) \right) \right) ≤
    \frac12\dim y - k <
    \frac12\dim x - k.
    \qedhere
    \]
\end{enumerate}
\end{proof}

\printbibliography

\end{document}
