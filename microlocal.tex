\documentclass[english,paper=letter,no-theorem-numbers]{short-notes}
\usepackage{math-ag}

\addbibresource{global.bib}
%\bibliography{global.bib}

\title{Microlocal definition of perverse coherent sheaves}
\author{Clemens Koppensteiner}

\begin{document}

\maketitle

The following is a proposed definition for \enquote{microlocal perversity} in sense of \cite[Definition~10.3.7]{KashiwaraSchapira:1994:SheavesOnManifolds}.

\section*{Definition}

Let $G$ act on a scheme $X$ such that all orbits have even dimension. 
We'll write $\cat D^b(X)$ for the derived category of equivariant sheaves with coherent cohomology.
We want to give a microlocal description of perverse sheaves in this category with respect to the \enquote{middle perversity}.
All categories of sheaves in the following should be interpreted as being the respective (bounded) derived categories.

For $\sheaf F ∈ \catCoh{X}$ let $\singsupp \sheaf F$ be defined as in Arinkin's talk\todo{Give that definition here.}.
Let $p ∈ H^{-1}(T^*X) ⊆ \Spec A[ξ₁,\dotsc,ξ_n]$ and define
\[
\cat D^b(X;p) = \rquot{\cat D^b(X)}{\left\{ \sheaf F ∈ \cat D^b(X) : p \notin \singsupp \sheaf F \right\}},
\]
where the notation $\cat C/N$ means the localization of $\cat C$ at the multiplicative system associated to the null-system $N$, i.e.~at
\[
\bigl\{ f\colon X → Y : \smash{\text{there is a d.t.~$X \xrightarrow{f} Y → Z \xrightarrow{+1}$ with $Z ∈ N$}} \bigr\}.
\]

Then we can make the following definition:
\begin{Def}
    Let $\perv[μ]\cat D^{≤0}(X)$ (resp.~$\perv[μ]\cat D^{≥0}(X)$) be the full subcategory of $\cat D^b(X)$ consisting of objects $\sheaf F$ such that:

    For every non-singular point $p ∈ \singsupp \sheaf F$ such that $\singsupp \sheaf F → X$ has constant rank in a neighborhood of $p$, there exists an equidimensional $G$-invariant subvariety $j\colon Y \hookrightarrow X$ and $\sheaf L ∈ \catPerf{Y}$ such that $\sheaf F \cong j_*\sheaf L[\frac12\dim Y]$ in $\cat D^b(X;p)$ and $H^j(\sheaf L) = 0$ for $j>0$ (resp.~$j<0$).
\end{Def}

\begin{Claim}
    $\perv[μ]\cat D^{≤0}(X) = \perv\cat D^{≤0}(X)$ and $\perv[μ]\cat D^{≥0}(X) = \perv\cat D^{≥0}(X)$, where $p$ is the middle perversity.
\end{Claim}

\section*{Discussion}

The claim is not true on $[\as2/\SL2]$.
For example, $\O_{(0)}[1] ∈ \perv[μ]\cat D(X)$, while $\O_{(0)}\oplus \O_{\as2}[1] \notin \perv[μ]\cat D(X)$.
The problem for $\O_{(0)}[1]$ is that $\O_{(0)}$ is perfect on $\as 2$.
The problem with the second sheaf is that its singular support $\as 2$ is not big enough to tell the two parts apart.

On the other hand, on $X=[N/\SL2]$, we have $\singsupp \O_N = N$, while $\singsupp \O_{(0)}$ is a \enquote{spoke} at $0$.
Also, the skyscraper is not perfect on $N$ anymore, so that the problems don't arise.

There seem to be two possible solutions:
\begin{enumerate}
    \item Find a better notion of singular support, that takes the \enquote{stackyness} into account.
        Additionally find a way around the problem that perfect sheaves on subvarieties might also be perfect on the whole variety.
    \item Restrict to cases with \enquote{enough} singularities.
\end{enumerate}

\printbibliography

\end{document}
